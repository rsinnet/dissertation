\documentclass[twocolumn]{article}

\usepackage{amsmath,amssymb,amsthm,mathrsfs}
\usepackage[top=1.25in, bottom=1.25in, left=.75in, right=.75in]{geometry}

\usepackage[english]{babel}
\usepackage{fancyhdr,lastpage}

\usepackage{../rsinnet}

\author{R.~W.~Sinnet}
\title{On the Stability of Energy Shaping}

\usepackage{times}

\pagestyle{fancy} \fancyhf{}
\lhead[]{On the Stability of Energy Shaping}
\rhead[]{R. W. Sinnet}
%%HASH%%
\cfoot[]{-\thepage\ of \pageref{LastPage}-}
\rfoot[]{Revised \today}

\begin{document}
\maketitle
\thispagestyle{fancy}

\section*{Overview}
The goal is to show that the energy shaping control law does not locally destabilize the system.
%
Formally, this means that, given an exponentially-stable limit cycle in a dynamical system, application of energy shaping results in a closed-loop system which is exponentially stable.

\section{Setup}

Consider a hybrid dynamical system with total energy $E(x)$.
%
The system has coordinates $x \in \D$ which take values in the {\em domain of admissibility}, $\D$.

The discrete aspect comes from a {\em guard constraint}, $h : \Q \to \R^{+}$, which leads to a transverse plane, $\S \subset \D$, called the {\em switching surface}.
%
The uncontrolled hybrid system can be written
\begin{align}
  \HS = \left\{
  \begin{array}{l l}
    {\dot x} = f(x), & x \in \D \setminus \S,\\
    x^{+} = \Delta(x^{-}), &x \in \S,
  \end{array}\right.
  \label{eq:hsys}
\end{align}
Under the action of control effort $u$, the corresponding hybrid control system has the form
\begin{align}
  \HCS = \left\{
  \begin{array}{l l}
    {\dot x} = f(x) + g(x) \, u, & x \in \D \setminus \S,\\
    x^{+} = \Delta(x^{-}), &x \in \S,
  \end{array}\right.
  \label{eq:csys}
\end{align}
for control values in a set $\U \subseteq \R^{m}$.
%
For the continuous dynamics, construct a Lyapunov function, $V : \D \to \R^{+}$, of the form
\begin{align}
  \label{eq:lyap}
  V(x) = \frac{1}{2} (E(x) - \Eref)^2.
\end{align}
Consider the energy shaping controller
\begin{align}
  \label{eq:es-qp} \tag {QP}
  \mueps(x) = \argmin_{u(x) \in \R^{n}} \ & u(x)^T u(x)\\
  \nonumber
  \mbox{s.t. } & L_{f} V(x) + L_{g} V(x) u(x) + \epsilon V(x) \leq 0.
\end{align}
Applying this to the system \eqref{eq:hsys} results in the closed-loop dynamics
\begin{align}
  \label{eq:hsys-cl}
  \HS = \left\{
  \begin{array}{l l}
    {\dot x} = f(x) + g(x) \, \mueps(x), & x \in \D \setminus \S,\\
    x^{+} = \Delta(x^{-}), &x \in \S.
  \end{array}\right.
\end{align}



\section{Stability of the Shaped System}

The main idea is stated as follows:
%
\begin{theorem}
  Given an exponentially-stable limit cycle in a hybrid dynamical system of the form \eqref{eq:hsys}, application of the energy shaping controller \eqref{eq:es-qp} to the control system \eqref{eq:csys} results in the closed-loop hybrid system \eqref{eq:hsys-cl}, which is exponentially stable.
\end{theorem}
%
The proof is given later.



\section{Equivalence of Invariant Orbits}

In order to achieve the stated goal, it is necessary to show that, given a system with a limit cycle representing the desired behavior, energy shaping can be applied and the resulting system will have an invariant orbit which is equivalent to the nominal system. Simply put, the control contribution from the energy shaping controller must be identically zero on the orbit. Consider the following lemma:

\begin{lemma}
  Applying the energy shaping controller \eqref{eq:es-qp} to a hybrid control system \eqref{eq:csys} results in a closed-loop system that demonstrates a periodic orbit which is identical to the unshaped system \eqref{eq:hsys}.
\end{lemma}

\begin{proof}
  For states on the periodic orbit, i.e., $\xst \in \orbit$, the energy is a known constant, $E(\xst) = E_{0}$.
  %
  Therefore, the limit cycle represents an invariant level set of the energy.
  %
  By construction of the Lyapunov function \eqref{eq:lyap} used in the the controller \eqref{eq:es-qp}, it is clear that $V(\xst) = 0$ and, moreover, that $$\min_{x \in \D} V(x) = 0.$$
  %
  The solution to the optimization problem \eqref{eq:es-qp} has cost $u(x)^T u(x) = 0$ (which implies that all elements of $u(x)$ are zero) and argument $u(x) = 0$ and this satisfies the stability condition of the control Lyapunov function; indeed ${\dot V}(\xst) = 0$ since the energy does not change without external forcing.
  %
  Thus, the periodic orbits are equivalent.
\end{proof}

\section{Zero Dynamics Formulation}

In order to understand the nature of energy shaping, consider breaking up the system into two sets of coordinates.
%
By considering level sets of energy, the continuous dynamics can be expressed as
\begin{align}
  \nonumber
  \dot \eta &= f(\eta, \zeta) + g(\eta, \zeta) \, u,\\
  \dot \zeta &= q(\eta, \zeta) + w(\eta, \zeta) \, u,
  \label{eq:zd-vfield}
\end{align}
%
with states $\eta \in \Eta$ and $\zeta \in \Zeta$ and control inputs $u \in \U$.
%
The vector fields $f$, $g$, $q$, and $w$ are assumed to be locally Lipschitz.
%
The natural choice of transformation to convert the continuous dynamics of \eqref{eq:csys} to \eqref{eq:zd-vfield} is through energy.
%
Thus consider the coordinate transformation
\begin{align}
  \eta := E(x) - \Eref,
\end{align}
which is a local diffeomorphism.
%
Applying the control law \eqref{eq:es-qp}, the dynamics becomes
\begin{align}
  \nonumber
  \dot \eta &= f(\eta, \zeta) + g(\eta, \zeta) \, \mueps(\eta, \zeta),\\
  \dot \zeta &= q(\eta, \zeta) + w(\eta, \zeta) \, \mueps(\eta, \zeta).
  \label{eq:zd-vfield-cl}
\end{align}

By the construction of the control law \eqref{eq:es-qp}, it is clear that $\mueps(0, \zeta) = 0$ and thus it follows that $f(0, \zeta) = 0$.
%
In other words, the zero dynamics manifold $\Zeta$ is the restricted subset of $X$ such that $\eta = 0$.
%
Consider the candidate Lyapunov function
\begin{align}
  V(\eta) = \frac{1}{2} \eta^{2}.
\end{align}

\begin{proposition}  
  Exponential stability of the $\eta$ dynamics is guaranteed if a {\em control Lyapunov function (CLF)} exists satisfying
  \begin{align}
    \label{eq:lyap-cond-nrg-es}
    &c_{1} \| \eta \|^{2} \leq \V(\eta) \leq c_{2} \| \eta \|^2,\\
    \nonumber
    &\inf_{u \in \U} \left[ L_{f} \V(\eta, \zeta) + L_{g} \V(\eta, \zeta) \, u + c_{3} \V(\eta) \right] \leq 0,
  \end{align}
  for all $(\eta, \zeta) \in \Eta \times \Zeta$.
\end{proposition}


\begin{proof}
  It is easy to see that the first inequality is satisfied for $c_{1} \leq \frac{1}{2}$ and $c_{2} \geq \frac{1}{2}$.
  %
  Define the set
  \begin{align}
    \K = \left\{ u \in \U : L_{f} \V(\eta, \zeta) + L_{g} \V(\eta, \zeta) \, u + c_{3} \V(\eta) \leq 0 \right\}.
  \end{align}
  For the CLF \eqref{eq:lyap-cond-nrg-es} with any locally Lipschitz feedback control law $u : \Eta \times \Zeta \to \K$, it follows that solutions satisfy
  \begin{align*}
    \| \eta(t) \| \leq \sqrt{\frac{c_{2}}{c_{1}}} e^{-\frac{c_{3}}{2} t} \| \eta(0) \|. & \qedhere
  \end{align*}
  
\end{proof}

Combine the continuous dynamics \eqref{eq:zd-vfield} with the reset map to obtain the hybrid control system
\begin{align}
  \label{eq:hcsys}
        \HCSbar= \left\{
        \begin{array}{l l}
          \begin{array}{r c l}
            {\dot \eta} &=& f(\eta, \zeta) + g(\eta, \zeta) \, u\\
            {\dot \zeta} &=& q(\eta, \zeta) + w(\eta, \zeta) \, u
          \end{array} & \mbox{ if } (\eta, \zeta) \in \D \setminus \S,\\
          \begin{array}{r c l}
            \eta^{+} &=& \Delta_{\eta}(\eta^{-}, \zeta^{-})\\
            \zeta^{+} &=& \Delta_{\zeta}(\eta^{-}, \zeta^{-})
          \end{array} & \mbox{ if } (\eta, \zeta) \in \S.
        \end{array}\right.
\end{align}

\section{Proof of Main Result}

First, let us define the domain and switching surface of the system:
\begin{align}
  \D &= \left\{ (\eta, \zeta) \in \Eta \times \Zeta : h(\eta, \zeta) \geq 0 \right\},\\
  \nonumber
  \S &= \left\{ (\eta, \zeta) \in \Eta \times \Zeta : h(\eta, \zeta) = 0 \mbox{ and } {\dot h}(\eta, \zeta) < 0 \right\}.
\end{align}

Now Theorem 1 can be proven:
\begin{proof}
  By the discrete converse Lyapunov theorem, exponential stability of $\Orbit$ implies the existence of a discrete Lyapunov function $\V : B_{\delta}(0) \cap \S \to \Rnn$ satisfying
  %
  \begin{align}
    \nonumber
    r_{1} \| x \|^{2} \leq \Vn(x) \leq r_{2} \| x \|^2,\\
    \Vn(\Pe(x)) - \Vn(x) \leq -r_{3} \| x \|^{2},
    \label{eq:lyap-cond-nom}
  \end{align}
  for some $r_{1}, r_{2}, r_{3} \in \Rnn$.
  %
  In addition, consider the CLF associated with \eqref{eq:es-qp} which is $\Ve : \X \to \Rnn < 0$.
  %
  Denote by $\Vee = \VeS$ the restriction of the CLF $\Ve$ to the switching surface $\S$.
  %
  Using these Lyapunov functions, define the candidate Lyapunov function
  \begin{align}
    \VP(x) = \Vn(x) + \sigma \Vee(x).
  \end{align}

  From \eqref{eq:lyap-cond-nom}, it is apparent that $\VP(x)$ is bounded as follows:
  \begin{align}
    \left( r_{1} + \sigma c_{1} \right) \| x \|^{2} \leq \VP(x) \leq \left( r_{2} + \sigma \frac{c_{2}}{\epsilon^{2}}  \right) \| x \|^{2}.
  \end{align}

  It must be shown that
  \begin{align}
    \VP(\Pe(x)) - \VP(x) \leq -\alpha \| x \|^{2}.
  \end{align}

  First note that
  \begin{align}
    \VP(\Pe(x)) - \VP(x) = \Vn(\Pe(x)) - \Vn(x) \hspace{1.2cm}\\
    \nonumber
    \mbox{} + \sigma(\Vee(\Pe(x)) - \Vee(x)).
  \end{align}

  Thus, it remains to be shown that
  \begin{align}
    \sigma(\Vee(\Pe(x)) - \Vee(x)) \leq -\beta \| x \|^{2}.
  \end{align}
  for $\beta \in \Rpd$.

\end{proof}

\end{document}
