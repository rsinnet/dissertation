\documentclass{article}

\usepackage{amsmath,amssymb}
\usepackage{fullpage}
\newcommand{\R}{\mathbb{R}}
\newcommand{\Q}{\mathcal{Q}}

\author{R.~W.~Sinnet}
\title{Energy of the Compass Gait}

\begin{document}
\maketitle

The total energy of the system is given by
\begin{align*}
  E(q, \dot q) = \frac{1}{2} {\dot q}^{T} M(q) {\dot q} + V(q).
\end{align*}
%
Impacts satisfy the following dynamics:
%
\begin{align*}
  \left[\begin{array}{c c}
      M(q^{-}) & -J^{T}(q^{-})\\
      J(q^{-}) & \mathbf{0}
    \end{array}\right]
  \left[\begin{array}{c}
      {\dot q}^{+}\\
      F_{\mathrm{imp}}
    \end{array}\right]
  = \left[\begin{array}{c}
      M(q^{-}) {\dot q}^{-}\\
      \mathbf{0}
      \end{array}\right]
\end{align*}
%
Under the assumption that the positions do not change through impact, the above reduces to
%
\begin{align*}
  \left[\begin{array}{c c}
      M(q) & J^{T}(q)\\
      J(q) & \mathbf{0}
    \end{array}\right]
  \left[\begin{array}{c}
      {\dot q^{+}}\\
      F_{\mathrm{imp}}
    \end{array}\right]
  = \left[\begin{array}{c}
      M(q) {\dot q}^{-}\\
      \mathbf{0}
    \end{array}\right].
\end{align*}
%
Using the Schur complement, the post-impact velocity can be expressed as a map $P : T\Q \to \R$ given by
\begin{align}
  \label{eq:reset-map}
  P(q, {\dot q}) = \left( I - M^{1}(q) J^{T}(q) \left( J(q) M^{-1}(q) J^{T}(q) \right)^{-1} J(q) \right) {\dot q}
\end{align}

In addition, positions on the guard must satisfy the constraint that
\begin{align}
  \label{eq:guard-constraint}
  \tan \gamma = \frac{p_{\mathit{nsf}}^{z}}{p_{\mathit{nsf}}^{x}}.
\end{align}

{\large Questions}
\begin{enumerate}
\item Is the image of the energy level set (corresponding to the fixed point) under the reset map invariant anywhere other than for the fixed point?
\item Do there exist other energy level sets which are invariant under the reset map?
\end{enumerate}

In order to obtain energy invariance through impact, the following condition must be satisfied:
\begin{align*}
  T(q, {\dot q}^{+}) - T(q^{-}, {\dot q}^{-}) = \Sigma m \ g h_{\mathit{nsf}}(q).
\end{align*}
where we assume that no adjustment is performed on the potential energy reference level.
%
Using \eqref{eq:reset-map}, we can write the energy-invariance relationship as
\begin{align}
  \label{eq:energy-conservation}
  \frac{1}{2} P^{T}(q, {\dot q}) M(q) P(q, \dot q) - \frac{1}{2} {\dot q}^{T} M(q) {\dot q} = \Sigma m \ g h_{\mathit{nsf}}(q).
\end{align}
%
It is important to note that $P(q, {\dot q})$ is affine in ${\dot q}$. This means the above equation can be solved rather easily for coordinates on an embedded submanifold.

In order to answer question 2, we must perturb along the codimension-2 manifold which results from the imposition of the two above constraints, namely, \eqref{eq:guard-constraint} and \eqref{eq:energy-conservation}.
\end{document}
