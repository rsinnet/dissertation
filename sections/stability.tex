\chapter{\uppercase{Lyapunov Stability}}

This chapter discusses stability of hybrid systems. Because of the unique nature
of hybrid systems, analysis of stability requires special treatment which is
distinct from continuous-time systems and more closely parallels that of
discrete-time systems.

\section{History of Lypaunov Stability}
\todo{Exposition of the development and some cases.}

\section{Stability of Continuous Systems}

In order to understand the theory presented in this work, it is necessary to
understand not only the stability of hybrid systems but also the stability of
continuous systems which partly comprise hybrid systems.
%
Contemporary treatment of stability generally relies on results initially
presented in the doctoral work of Aleksandr Lyapunov in 1892 \cite{}.

\subsection{Stability about an Equilibrium Point}
This work is chiefly concerned with autonomous systems of the form
\begin{align}
  \label{eq:autsys}
  \dx = \xf\arx
\end{align}
where $f : \D \to \Rn$ is a locally Lipschitz vector field valid on some domain
$\D \subset \Rn$.
%
Without loss of generality, suppose that the system \eqref{eq:autsys} has an equilibrium point at the
origin;
%
in other words, $\xf\argbzero = \boldzero$.
%
\begin{definition}
  The equilibrium point $\x = \boldzero$ is stable if, for each $\epsilon > 0$,
  $\exists \delta > 0$ such that
  \begin{align*}
    \nxzero < \delta \Rightarrow \nxt < \epsilon \ \forall \ t
    \geq 0.
  \end{align*}
  The equilibrium point is unstable if it is not stable.
\end{definition}

In addition to the above definition, there are stricter forms of stability:
\begin{definition}
  The equilibrium point $\x = \boldzero$ is {\bf asymptotically stable} if it is
  stable and $\delta$ can be chosen such that
  \begin{align*}
    \nxzero < \delta \Rightarrow \lim_{t \to \infty} \x\argt = 0.
  \end{align*}
\end{definition}

Further, an even stronger form of stability is of interest in this work:
\begin{definition}
  The equilibrium point $\x = \boldzero$ is {\bf exponentially stable} if there
  exist real constants $c, k, \lambda > 0$ such that
  \begin{align*}
    \nxt \leq k \nxtnaught e^{-\lambda (t - t_{0})} \ \forall \ \nxtnaught
    < c.
  \end{align*}
\end{definition}


\section{Stability of Discrete Systems}

In analogy to continuous systems, consider the autonomous system which satisfies
the difference equation
\begin{align}
  \label{eq:disc-sys}
  x_{k+1} = f(x_{k}).
\end{align}
where $f : \D \to \D$ is a locally Lipschitz vector field valid on some domain
$\D \subset \Rn$.
%
Without loss of generality, suppose that the system \eqref{eq:disc-sys} has an
equilibrium point at the origin;
%
in other words, $f\argbzero = \boldzero$.
%
\begin{definition}
  The equilibrium point $\x = \boldzero$ of \eqref{eq:disc-sys} is stable if, for each $\epsilon > 0$,
  $\exists \delta > 0$ such that
  \begin{align*}
    \nxkzero < \delta \Rightarrow \nxk < \epsilon \ \forall \ k
    \geq 0.
  \end{align*}
  The equilibrium point is unstable if it is not stable.
\end{definition}

In addition to the above definition, there are stricter forms of stability:
\begin{definition}
  The equilibrium point $\x = \boldzero$ of \eqref{eq:disc-sys} is {\bf asymptotically stable} if it is
  stable and $\delta$ can be chosen such that
  \begin{align*}
    \nxkzero < \delta \Rightarrow \lim_{k \to \infty} \xk = 0.
  \end{align*}
\end{definition}

Further, an even stronger form of stability is of interest in this work:
\begin{definition}
  The equilibrium point $\x = \boldzero$ of \eqref{eq:disc-sys} is {\bf
    exponentially stable} if there exist real constants $\alpha, \beta, c> 0$
  such that, for all $\nxknaught < c$ and $k \geq k_{0}$,
  \begin{align*}
    \nxk \leq \beta \nxknaught^{\alpha}.
  \end{align*}
\end{definition}



\begin{enumerate}
\item Lyapunov theorem for discrete systems
\item Proof
\end{enumerate}