\chapter{\uppercase{Lyapunov Stability}}

This chapter discusses notions of stability in hybrid dynamical systems.
%
Because of the unique nature of hybrid systems, analysis of stability requires
special treatment which is distinct from continuous-time systems and more
closely parallels that used for discrete-time systems.
%
Stability of dynamical systems has a long history, appearing in the works of
researchers such as Routh \cite{Routh1877} and Hurwitz \cite{Hurwitz1895}.
%
Contemporary notions of stability in nonlinear systems generally rely on results
initially presented in the doctoral work of Aleksandr Lyapunov in his doctoral
thesis in 1892 \cite{Lyapunov1992}.
%
In said treatise, Lyapunov described two methods for analyzing the stability of
equilibrium points of ordinary differential equations.

The {\em first method}, sometimes called the {\em indirect method of Lyapunov},
provides a means for understanding the stability properties of an equilibrium
point by examining a linearization of the nonlinear system.
%
Due to the nature of linearization, the results of such an analysis only pertain
locally and within an unknown region about the equilibrium point.
%
Nonetheless, this method is well-known and sees widespread use due to its
simplicity and straightforward nature.
%

The {\em second method}, sometimes called the {\em direct method of Lyapunov}
involves the use of scalar-valued functions of the state of a system called {\em
  Lyapunov functions} which satisfy specific conditions that are set forth
later.
%
Through the use of these Lyapunov functions, it is possible to prove
stability not only locally but in a known region containing an equilibrium point
or even globally.
%
These functions are not unique and a major drawback to this method is the lack
of an algorithmic procedure for constructing valid Lyapunov functions.

Both methods are used in this work but for different purposes:
%
Lyapunov's indirect method is often used to analyze the stability of hybrid
systems by examining the stability of a linearization of the \Poincare{} map
about the equilibrium point; this will be explained in greater detail later in
this chapter.
%
Accordingly, the numerical simulations in this work will rely on this usage of
Lyapunov's indirect method.
%
In order to formally demonstrate the stability of energy shaping---the main
focus of this work---Lyapunov's direct method will be employed.
%
This method is often used in theoretical constructions and can also be used to
understand domain of attraction, although the particular usage will preclude
this type of application.

Lyapunov's work on the direct method established sufficient conditions for
stability but lacked the notion of {\em uniform stability} which was required
for establishing necessary conditions.
%
The results presented by Lyapunov lay essentially dormant for decades until
researchers began to investigate the ideas further.
%
In the 1930's, researchers established minor results expanding on Lyapunov's
ideas.
%
Khalikoff \cite{Khalikoff1937} and Malkin \cite{Malkin1938} proved additional
theorems on stability which could be used to show stability in the sense of
Lyapunov with relaxed assumptions.
%
Masera \cite{Massera1949} provided more restrictive definitions, introducing the
notion of {\em equiasymptotic  stability}.
%
Yet it wasn't until the assumptions of uniformity were formulated by Malkin
\cite{Malkin1954} that the necessary framework existed in which to formulate
converse theorems.
%
Barbashin and Krasovskii further strengthed Malkin's result in \cite{Barbashin1954}.
%

After these observations were published, converse theorems followed shortly
thereafter thanks to researchers such as Kurzweil \cite{Kurzweil1956} and
Massera \cite{Massera1956}.
%
Converse theorems have seen substantial development and broad use since these results.
%
Hoppensteadt \cite{Hoppensteadt1966} presented constructions for singularly
perturbed systems in which he constructed a $C^{1}$ Lyapunov function, extending
existing work on singularly perturbed systems to unbounded time intervals.
%
Wilson \cite{Wilson1969} constructed a $C^{\infty}$ Lyapunov function for a
continuous vector field having an asymptotically stable invariant set.
%
Early results on converse Lyapunov functions for stability of sets are
summarized in numerous texts; see, e.g, \cite{Antosiewicz1958,Yoshizawa1975}.

In addition to providing necessary and sufficient conditions for stability,
researchers have also studied what conditions are necessary for solutions to be
integrable for all time.
%
Cesari provides conditions  for \cite[\S 1.5]{Cesari1971} for second-order
linear systems.
%
Strauss introduces $L^{p}$ stability to attempt to provide conditions for more
general systems in \cite{Strauss1965}.

As LaSalle points out \cite{LaSalle1964}, before the 1960's much of the work
done in the USSR was inaccessible to English-speaking audiences, but as the
decade progressed, this language barrier gradually dissipated.
%
An early text by LaSalle and Lefschetz (the first such text in English)
\cite{LaSalle1961} outlining the methods of Lyapunov stability theory contains
proofs which are accessible to those with less extensive mathematical
backgrounds.
%
Shortly thereafter, additional texts emerged including those of Krasovskii
\cite{Krasovskii1963} and of Hahn \cite{Hahn1967}.

Additional information on the history of Lyapunov theory can be found throughout
the literature; see, e.g., \cite{Michel2007,Teel1999}.
%
Definitions of stability are ubiquitous in the literature; see, e.g.,
\cite{Khalil2002,Teschl2012,Vidyasagar1993}.

\section{Stability of Continuous Systems}

In order to understand the theory presented in this work, it is necessary to
understand not only the stability of hybrid systems but also the stability of
continuous systems and discrete systems which partly comprise hybrid systems.
%

\subsection{Stability about an Equilibrium Point}
This work is chiefly concerned with autonomous systems of the form
\begin{align}
  \label{eq:autsys}
  \dx = \xf\arx
\end{align}
where $\xf : \D \to \Rn$ is a locally Lipschitz vector field valid on some domain
$\D \subset \Rn$.
%
Without loss of generality, suppose that the system \eqref{eq:autsys} has an equilibrium point at the
origin;
%
in other words, $\xf\argbzero = \boldzero$.
%
\begin{definition}
  The equilibrium point $\x = \boldzero$ is stable if, for each $\epsilon > 0$,
  $\exists \delta > 0$ such that
  \begin{align*}
    \nxzero < \delta \Rightarrow \nxt < \epsilon \ \forall \ t
    \geq 0.
  \end{align*}
  The equilibrium point is unstable if it is not stable.
\end{definition}

In addition to the above definition, there are stricter forms of stability:
\begin{definition}
  The equilibrium point $\x = \boldzero$ is {\bf asymptotically stable} if it is
  stable and $\delta$ can be chosen such that
  \begin{align*}
    \nxzero < \delta \Rightarrow \lim_{t \to \infty} \x\argt = \boldzero.
  \end{align*}
\end{definition}

Further, an even stronger form of stability is of interest in this work:
\begin{definition}
  The equilibrium point $\x = \boldzero$ is {\bf exponentially stable} if there
  exist real constants $c, k, \lambda > 0$ such that
  \begin{align*}
    \nxt \leq k \nxtnaught e^{-\lambda (t - t_{0})} \ \forall \ \nxtnaught
    < c.
  \end{align*}
\end{definition}


\section{Stability of Discrete Systems}

In analogy to continuous systems, consider the autonomous system which satisfies
the difference equation
\begin{align}
  \label{eq:disc-sys}
  x_{k+1} = f(x_{k}).
\end{align}
where $f : \D \to \D$ is a locally Lipschitz vector field valid on some domain
$\D \subset \Rn$.
%
Without loss of generality, suppose that the system \eqref{eq:disc-sys} has an
equilibrium point at the origin;
%
in other words, $f\argbzero = \boldzero$.
%
\begin{definition}
  The equilibrium point $\x = \boldzero$ of \eqref{eq:disc-sys} is stable if, for each $\epsilon > 0$,
  $\exists \delta > 0$ such that
  \begin{align*}
    \nxkzero < \delta \Rightarrow \nxk < \epsilon \ \forall \ k
    \geq 0.
  \end{align*}
  The equilibrium point is unstable if it is not stable.
\end{definition}

In addition to the above definition, there are stricter forms of stability:
\begin{definition}
  The equilibrium point $\x = \boldzero$ of \eqref{eq:disc-sys} is {\bf asymptotically stable} if it is
  stable and $\delta$ can be chosen such that
  \begin{align*}
    \nxkzero < \delta \Rightarrow \lim_{k \to \infty} \xk = \boldzero.
  \end{align*}
\end{definition}

Further, an even stronger form of stability is of interest in this work:
\begin{definition}
  The equilibrium point $\x = \boldzero$ of \eqref{eq:disc-sys} is {\bf
    exponentially stable} if there exist real constants $\alpha, \beta, c> 0$
  such that, for all $\nxknaught < c$ and $k \geq k_{0}$,
  \begin{align*}
    \nxk \leq \beta \nxknaught^{\alpha}.
  \end{align*}
\end{definition}



\begin{enumerate}
\item Lyapunov theorem for discrete systems
\item Proof
\end{enumerate}