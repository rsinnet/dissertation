\chapter{\uppercase{Model Identification Methods}}

The basic problem of system identification boils down to using measured data
along with a model of appropriate complexity, typically a gray-box model, to
determine parameters for a system which can accurately predict a system's
dynamics with respect to observable behaviors of the system.
%
This can be accomplished in a number of ways and, given accurate and
reasonably unbiased sensing information, can be result in models which
account for a very high percentage of the measurement data.
%

\section{Fundamentals}
When viewed in a certain light, all model identification methods for mechanical
systems involve identifying the fundamental relationship between force and
acceleration described by Newton's Second Law (see \cite{feynman1963}), namely,
that
\begin{align}
  \label{eq:newtons-second}
  F = m a,
\end{align}
which expresses the basic property of physics that applying a force to a massive
object induces a particular acceleration.
%
Through the property of superposition, this relationship eventually leads to
standard dynamic model of a robot as expressed in \ref{eq:robot-dynamics}, but
at its core, model identification is merely using this relationship to identify
a system's mass configuration.

Like most problems, there is a variety of approaches, each of which has its own
merits.
%
Perhaps one of the more straightforward of the methods leverages the fact that
the dynamics of a rigid kinematic chain is a affine with respect to certain
inertial parameters such as the math.
%
This leads to a relationship of the form
\begin{align}
  \label{eq:regressor}
  \tau = \regressorargs \regressorparam
\end{align}
where $\tau$ represents the experienced torque, $\regressorargs$ is known as the
{\em regressor} (see \cite{spong1989}), and $\regressorparam$ represents a set
of parameters, sometimes called the {\em base parameter set}.

\section{Offline Identification}

\section{Online Identification}


\begin{enumerate}
  \item Offline vs. Online -- advantages of each
  \item Sources of error
\end{enumerate}

%%%%%%%%%%%%%%%%%%%%%%%%%%%%%%%%%%%%%%%%%%%%%%%%%%%%%%
%\begin{figure}[H]
%\centering
%\includegraphics[scale=.50]{figures/Penguins.jpg}
%\caption{TAMU figure}
%\label{fig:tamu-fig3}
%\end{figure}
%%%%%%%%%%%%%%%%%%%%%%%%%%%%%%%%%%%%%%%%%%%%%%%%%%%%%%
%\section{Another Section}


%%%%%%%%%%%%%%%%%%%%%%%%%%%%%%%%%%%%%%%%%%%%%%%%%%%%%%%
%\begin{figure}[H]
%\centering
%\includegraphics[scale=.50]{figures/Penguins.jpg}
%\caption{Another TAMU figure}
%\label{fig:tamu-fig4}
%\end{figure}
%%%%%%%%%%%%%%%%%%%%%%%%%%%%%%%%%%%%%%%%%%%%%%%%%%%%%%%

%\subsection{Subsection}

%%%%%%%%%%%%%%%%%%%%%%%%%%%%%%%%%%%%%%%%%%%%%%%%%%%%%%%
%\begin{figure}[H]
%\centering
%\includegraphics[scale=.50]{figures/Penguins.jpg}
%\caption{Another TAMU figure}
%\label{fig:tamu-fig4-2}
%\end{figure}
%%%%%%%%%%%%%%%%%%%%%%%%%%%%%%%%%%%%%%%%%%%%%%%%%%%%%%%
%\subsection{Subsection}

%A table example is going to follow.

%\begin{table}[H]
%\centering
%\caption{This is a table template}
%\begin{tabular}{|l|c|c|c|c|c|}
%\hline
%Product & 1 & 2 & 3 & 4 & 5\\
%\hline
%Price & 124.- & 136.- & 85.- & 156.- & 23.-\\
%Guarantee [years] & 1 & 2 & - & 3 & 1\\
%Rating & 89\% & 84\% & 51\% & & 45\%\\
%\hline
%\hline
%Recommended & yes & yes & no & no & no\\
%\hline
%\end{tabular}
%\label{tab:template2}
%\end{table}
%\section{Another Section}
