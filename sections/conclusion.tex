\chapter{\uppercase{Conclusion}}

This thesis presented formal results for the method of energy shaping in
mechanical systems.
%
By taking advantage of conservation of energy, energy shaping seeks to improve
the performance characteristics of existing periodic behaviors.
%
Through the use of control Lyapunov functions, energy shaping acts on a
non-conservative system while guaranteeing that the periodic behaviors are not
destabilized.
%
This formal guarantee is a novel contribution that provides a significant
improvement over existing results which lacked this consideration.

As demonstrated, the methods, in some cases, may increase the domain of
attraction thereby increasing robust with respect to perturbations in initial
condition.
%
Simulation results also demonstrated that energy shaping appears to drive
systems to faster convergence.
%%
The method was demonstrated on simple models to provide intuition and then on
more complex models to show versatility.
%
Simulations results and modeling for a multi-phase human gait for a biped with
feet were also provided to demonstrate energy shaping on a multi-domain hybrid
system.

When proving energy shaping, it was shown that a Lyapunov function exists
locally in the \Poincare{} map.
%
The exact form of this Lyapunov function is not known, but this begs the
question:
%
is it possible to explicitly construct a Lyapunov function, perhaps by guessing,
that could be used to analytically examine the domain of attraction?
%
In doing so, one would likely obtain a more complete understanding of robustness
properties.
%

In addition, this thesis does not make any claims about robustness with respect
to model uncertainties and measurement uncertainties.
%
It would be worthwhile to investigate model and measurement robustness in the
context of energy shaping and control Lyapunov functions in a wider context.
%
Such discussions are spread throughout the literature; see, e.g.,
\cite{Freeman1996}.
%
Open questions about the domain of attraction and robustness are pervasive
throughout the literature and are not unique to energy shaping;
%
however, their answers are left as the subject of future exploration.

%{\bf Need to tie in human-inspired control.}
