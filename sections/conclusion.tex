\chapter{\uppercase{Conclusion}}

This thesis presented formal results for the method of energy shaping in
mechanical systems.
%
By taking advantage of conservation of energy, energy shaping seeks to improve
the performance characteristics of existing periodic behaviors.
%
Through the use of control Lyapunov functions, energy shaping acts on a
non-conservative system while guaranteeing that the periodic behaviors are not
destabilized.
%
This formal guarantee is a novel contribution that provides a significant
improvement over existing results which lacked this consideration.


As demonstrated, the methods, in some cases, may increase the domain of
attraction thereby increasing robust with respect to perturbations in initial
condition.
%
Simulation results also demonstrated that energy shaping appears to drive
systems to faster convergence.
%
The method was demonstrated on simple models to provide intuition and then on
more complex models to show versatility.
%
Simulations results and modeling for a multi-phase human gait for a biped with
feet were also provided to demonstrate energy shaping on a multi-domain hybrid
system.


When proving energy shaping, it was shown that a Lyapunov function exists
locally in the \Poincare{} map.
%
The exact form of this Lyapunov function is not known, but this begs the
question:
%
is it possible to explicitly construct a Lyapunov function, perhaps by guessing,
that could be used to analytically examine the domain of attraction?
%
In doing so, one would likely obtain a more complete understanding of robustness
properties.
%
Despite the lack of formal claims with respect to global properties, the local
stability properties are practically useful when using energy shaping as a
stabilizing controller in operating regions around the desired behavior.


In addition, this thesis does not make any claims about robustness with respect
to model uncertainties and measurement uncertainties.
%
It would be worthwhile to investigate model and measurement robustness in the
context of energy shaping and control Lyapunov functions in a wider context.
%
Such discussions are spread throughout the literature; see, e.g.,
\cite{Freeman1996}.
%
Open questions about the domain of attraction and robustness are pervasive
throughout the literature and are not unique to energy shaping;
%
however, their answers are left as the subject of future exploration.


When it comes to implementing formal algorithms on cyber-physical systems, there
is a number of pertinent considerations which determine the viability and
efficacy of a given controller.
%
A control algorithm which updates its commands too slowly will cause shaky
motion which in itself is indicative of imprecise control.
%
In addition, a slow loop rate means the system is slow to react to external
disturbances which makes it more difficult to reject them, thereby decreasing
the robustness of the system.
%
With bipeds this concern is of the utmost importance as the stability of a biped
depends on balancing the dynamic moment to keep the robot from toppling over.
%
Because feet tend to be narrow, stability in the coronal plane is a challenge
that frequently plagues roboticists.


In order to achieve smooth behavior in dynamic robots, it is therefore necessary
to have a relatively fast control loop and thus researchers seek to design
controllers which are computationally tractable.
%
Nonlinear controllers often give superior behavior in simulation but
their complexity renders them impractical for real-time implementation.
%
Oftentimes, controllers for robots are formulated as optimization problems which
use internal models to determine the appropriate control values.
%
The idea is to constrain the optimization problem to achieve the desired
behavior; typical formulations might take into consideration control of the zero
moment point to guarantee the system remains standing.


Considerations such as these generally lead to nonlinear optimization problems
which are problematic for a variety of reasons.
%
While a complex nonlinear formulation may represent a rigorous model of known
physical phenomena and may therefore match up very well with reality, the
computation cost may be prohibitive rendering such a formulation useful only in
simulation.
%
To overcome this problem, some researchers are turning quadratic programs for
which there exist efficient algorithms to quickly find solutions.
%
If one can formulate the constraints as a convex set, the solution can be found
easily and one can guarantee optimality whereas with nonlinear optimizations,
there is no guarantee on optimality over a given region.


The formulation presented in this paper is in the form of a quadratic program
whose optimization variables are joint torques and ground reaction wrenches.
%
This has great value because one can approximate friction cones with friction
pyramids (see \cite{Grizzle2014}) which are convex sets rendering friction
constraints amenable to the optimization problem.
%
Since the zero-moment point has a linear dependence on joint torques and ground
reaction wrench, this can also be factored into a quadratic program.
%
In addition, it is straightforward to impose bounds on torque.
%
Any of the above constraints or their combination may destroy the
feasibility of the optimization posed in \eqref{eq:es-qp}.
%
This potential issue with feasibility, however, can be addressed by relaxing the
constraints of the control Lyapunov function although this mitigates the formal
guarantee on stability.
%
Yet this leads to an interesting question:
%
is it possible to relax the constraints of a control Lyapunov function given in
\defref{def:lyap-func} and still guarantee stability?


A recent trend has found researchers formulating model predictive controllers
which attempt to determine control values by predicting their effect on the
behavior of the system over a future time window.
%
Energy shaping fits right into model predictive control (MPC) and could be used
to provide controllers which exhibit the benefits of energy shaping.
%
With MPC in particular, it is extremely important to have tractable controllers
as such controllers require multiple evaluations as the algorithm must integrate
forward to solve the relevant optimization problem.
%
The growth of computers is changing this paradigm but effective and robust
bipeds generally have fast control loops and this trend will likely continue for
some time.
%


Another area of interest which is seeing recent growth is precision control of
series elastic actuators (SEA), particularly in systems such as bipeds which
experience a wide range of dynamic forces.
%
While control of SEA is more challenging than control of traditional actuators,
the potential benefits make it an important phenomenon to understand.
%
The addition of series compliance not only allows for force sensing at joints
but also for energy storage which can be utilized through design methods like
energy shaping.
%
Force sensing leads to additional possibilities such as online system
identification which can provide vital information about the mass distribution
of a robot.
%
By improving the internal model, model-based methods such as energy shaping see
improved robustness; this is particularly helpful for MPC.
%

The models studied in this paper involve traditional actuation schemes rather
than SEA.
%
This decision choice was instrumental in achieving formal results but it is
important to recognize that SEA is seeing increasing use in robot design and
this trend is likely to continue.
%
In order to handle SEA formally, new control schemes have to be formulated which
take the compliance into account.
%
This naturally begs the question:
%
Can energy shaping be formulated in a tractable way on systems with SEA?
%
The answer to this question is not likely be straightforward as SEA requires the
addition of double integrators in the model which would change the nature of an
optimization problem like \eqref{eq:es-qp}.

