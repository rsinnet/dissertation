\chapter{\uppercase{Energy Shaping}} \label{ch:energy-shaping}

This chapter focuses on the primary contribution of this thesis: stabilization
 of periodic behaviors in hybrid dynamical systems through {\em energy shaping}.
%
This method provides a manner for understanding the dynamics of both
conservative and  non-conservative systems in terms of energy exchange and
employs Lyapunov methods to alter the stability properties of controllers for
such systems.
%
The formal theoretical contribution appears in \thmref{theorem:main-theorem}
which states that the energy shaping controller presented does not destabilize
the system to which it is applied for small enough control gains.
%
Although nothing is formally shown regarding the robustness of the shaped system
to perturbations in initial conditions, myriad numerical simulations later show
that, the method results in a larger domain of attraction and shorter
stabilization times for limit cycles.

It should be mentioned that this problem falls under a class of problems
involving stability of systems with zero dynamics.
%
In \cite{Ames2014}, a similar problem was considered in which a stabilizing
control law was constructed using control Lyapunov functions to stabilize to a
zero dynamics which exhibited hybrid invariance;
%
that is, for initial conditions on the intersection of the switching surface and
the hybrid zero dynamics manifold, application of the reset map will result in a
state which is still on the hybrid zero dynamics.
%
This was a key assumption underlying the \cite{Ames2014} but this assumption
does not hold for energy shaping as energy is generally not invariant through
impact, though there may be pathological examples which demonstrate this
property.
%
In fact, for certain conservative systems like the compass gait biped shown in
\figref{fig:compass_gait_slope_model}, which exhibits local exponential
stability, energy change can only occur through discrete transitions and so
impacts actually act as a stabilizing influence.

Energy shaping has been considered for conservative systems \cite{Spong2003} and
for non-conservative systems \cite{Spong2007}.
%
There is, however, a dearth of formal results and this thesis aims to fill in
some of the gaps in the literature.
%
In \cite{Spong2003}, Spong and Bhatia presented the idea of designing a
controller to shape the total energy of the system for the compass-gait biped
and showed that the controller would guarantee asymptotic stability to the
desired energy level through the continuous dynamics.
%
Later in \cite{Spong2007}, the results were extended to non-conservative systems
by considering storage functions of the form shown in
\eqref{eq:storage-function-Fnc}.
%
The exposition was generally based around numerical simulations and had limited
formal results.

\begin{figure}
  \centering
  \def\svgwidth{.5\columnwidth}
  \input{figs/cg2d-slope-model.eps_latex}
  \caption{Compass-gait biped falling down a slope.}
  \label{fig:compass_gait_slope_model}
\end{figure}

Another key difference is the assumption of stability which is necessary for the
application of energy shaping.
%
Whereas \cite{Ames2014} requires stability of the system for states restricted
to the hybrid zero dynamics, this work requires stability of the nominal system
and does not require hybrid invariance of the zero dynamics.
%
Thus, while the problems are somewhat similar, they also have their differences
and are applicable to different types of systems and problems.

\section{Overview}

Periodic behaviors in mechanical systems have specific energy signatures that
evolve over time.
%
The energy dynamics of a conservative system can be understood as an interplay
between kinetic and potential energy, with energy transfer occurring between
these two quantities.
%
Indeed for conservative systems, the energy is constant, i.e.,
\begin{align*}
  E_{0} \equiv E\argsqdq = T\argsqdq + U\argsq,
\end{align*}
where $T : T\Q \to \Rnn$ represents the kinetic energy and $U : \Q \to \R$
represents the potential energy.
%
In a non-conservative system, this relationship does not hold, but the dynamics
of energy can still be understood as
\begin{align}
  \label{eq:ncsys-nrg-cons}
  E_{0} \equiv \Ec\argt = T\argsqdqt + U\argsqt - \W\argt.
\end{align}
with the storage function $\W : \Rnn \to \R$ defined by
\begin{align}
  \label{eq:storage-function-Fnc}
  \W\argt = \int_{0}^{t} \! \Fnc\argtau \cdot \frac{d\q\argtau}{d\tau} \ d\tau
\end{align}
where $\Fnc: \Rnn \to \R^{n}$ represents non-conservative forcing.
%
From the above definition it is clear that the storage function is dependent on
initial conditions.
%
This work will consider forcing that takes the form of an autonomous feedback
control law, i.e.,
\begin{align*}
  \Fnc\argt = \Fnca\argsqdqt
\end{align*}
for autonomous forcing
\begin{align*}
  \Fnca\argsqdq := \B\argsq \, \vv\argsqdq.
\end{align*}
%
While the state variables $\q$ and $\dq$ vary with time, the forcing is
autonomous in the normal sense of the term in that the forcing does not depend
on the value of an independent variable time variable directly but rather only
on the values of $q$ and $\dq$ at any given point in time.
%
Time can be shifted arbitrarily and the resulting evolution of coordinates will
not changed.
%
For more on the difference between autonomous and non-autonomous (time-based)
systems, see, for example, \cite[pp.~63]{Perko2001}.

%
Under such an assumption, \eqref{eq:storage-function-Fnc} becomes
\begin{align}
  \label{eq:storage-function}
  W\argt = \int_{0}^{t} \! \B\argsqtau \, \vv\argsqdqtau \cdot
  \frac{dq\argtau}{d\tau} \ d\tau.
\end{align}

From the relationship stated in \eqref{eq:ncsys-nrg-cons} and
\eqref{eq:storage-function-Fnc}, it is clear that the total energy present in
the system plus the energy removed by non-conservative forcing is constant.
%
In other words, the integral term represents the flow of energy into and out of
the system.
%
This phenomenon is well understood and it holds for all motion in mechanical
systems, not just for periodic behaviors.
%
This storage function is central to passivity-based control, a topic which
has appeared previously in the literature; e.g., \cite{Spong2007}.

Understanding the energy dynamics of a system is key to the method of energy
shaping.
%
By taking advantage of the presence of energy level sets in periodic behaviors,
energy shaping seeks to augment certain types of controllers which do not
necessarily have an intrinsic notion of stabilizing to a zero dynamics as in
\cite{Ames2014}.
%
In doing so, the goal is to alter the properties of robustness with respect to
perturbations in initial conditions for existing controllers.
%
The design of such a controller will therefore seek to satisfy the above goals
and formal proof will be given that the designed controller does not destabilize
the existing behavior (induced by an existing control law).

\section{Setup}

The main idea behind energy shaping is to impose exponential convergence to a
conserved energy function through the continuous dynamics of a hybrid system of
the form \eqref{eq:hcsys} with the goal of not destabilizing the overall hybrid
system.
%
As mentioned in \chapref{ch:stability}, there are multiple types of stability
depending upon the type convergence exhibited by a system.
%
Energy shaping shaping seeks to achieve exponential convergence to a desired
energy level;
%
formally, given a conserved energy function $\Ec$ as in
\eqref{eq:ncsys-nrg-cons}, the following condition is desired:
\begin{align*}
  \Ec(t) \leq \Ec(0) \, e^{-\alpha t}
\end{align*}
for some positive real constant $\alpha$ and for $t \geq 0$.

For non-conservative systems, as mentioned above, a storage function is
necessary to track energy flow into and out of the system due to
non-conservative forcing as doing so allows energy shaping to take advantage of
conservation of energy.
%
The storage function that is of interest was described in
\eqref{eq:storage-function} but the evolution of the function must be expressed
as a differential equation for the dynamics to be properly understood.
%
By the fundamental theorem of calculus, it is clear that the desired storage
function must satisfy
\begin{align*}
  d\W = R\argsqdq \, dt
\end{align*}
where $R$ is given by
\begin{align*}
  R\argsqdq := \left(\B\argsq \, \vv\argsqdq\right)^{T} \frac{d\q}{dt}.
\end{align*}
%
Using this storage function $\W$, augment the state space viz.
\begin{align*}
  \x := \argsqdqW.
\end{align*}
%
In addition, the vector fields must be augmented to take the $\W$ dynamics into
account as follows:
\begin{align*}
  \xfbar\argsqdq := \left(\begin{array}{c}
      \xf\argsqdq + \xg\argsqdq \, \vv\argsqdq\\
      R\argsqdq
    \end{array}\right), &&
  \xgbar\argsq := \left(\begin{array}{c}
      \xg\argsq\\
      \boldzero
    \end{array}\right).
\end{align*}
Using the augmented state leads to the followed definition of an augmented
hybrid control system:
\begin{align}
  \HCSbar = \left\{
    \begin{array}{l c l}
      \begin{array}{r c l}
        \hspace{1.15em}\dx &=& \xfbar\argsqdq + \xgbar\argsq \, \uu
      \end{array} &
      \left.\vphantom{\begin{array}{r c l}
            \hspace{1.15em}\dx &=& \xfbar\argsqdq + \xgbar\argsq \, \uu
          \end{array}}\right\} & \mbox{if } \argsqmdqm \in \D \setminus \Guard,\\
      % 
      \begin{array}{r c l}
          \qp &=& \Deltaq\argsqm\\
          \dqp &=& \Deltadq\argsqmdqm\\
          \Wp &=& \DeltaW = 0
        \end{array} &
        \left.\vphantom{\begin{array}{r c l}
          \qp &=& \Deltaq\argsqm\\
          \dqp &=& \Deltadq\argsqmdqm\\
          \Wp &=& \DeltaW = 0
        \end{array}}\right\} & \mbox{if } \argsqmdqm \in \Guard.
    \end{array}\right.
  \label{eq:hcs_bar}
\end{align}
where $\Deltaq$, $\Deltadq$, and $\DeltaW$, respectively, represent the $\q$, $\dq$, and
$\W$ components of the reset map, $\Delta$.
%
Under this formulation, the conserved energy can be expressed as
\begin{align*}
  \Ec\argsqdqW = T\argsqdq + U\argsq - W.
\end{align*}
%
In this form, it is clear that the system of interest is an autonomous one.
%
This formulation will be used to state and prove the energy shaping theorem (\thmref{theorem:main-theorem}). 
%
However, before stating the theorem, it is important to first introduce and
discuss a tool which will play a crucial part in achieving energy stabilization:
the control Lyapunov function.

\section{Control Lyapunov Functions}

The energy shaping controller relies on the use of {\em control Lyapunov
  functions} \cite{Artstein1983,Freeman1996}, which are a natural extension of
Lyapunov's direct method.
%
Recall that a system of the form
\begin{align}
  \label{eq:csys-es}
  \dx = \xf\arx
\end{align}
is stable about the origin if there exists a function $\V : \X \to \Rnn$ that is
positive definite and zero at the origin.
%
Stability can be achieved about any given equilibrium point but a coordinate
transformation can be constructed to shift any point to the origin without loss
of generality.
%
For a system with no control (or a closed-loop system) of the form
\eqref{eq:csys-es}, Lyapunov stability can formally be defined:
%
\begin{definition}
  \label{def:lyap-func}
  For a system of the form \eqref{eq:csys-es}, a continuously differentiable
  function $V : \D \to \Rnn$ is said to be a {\bf \em Lyapunov function} if
  there exist constants $c_{1}, c_{2}, c_{3} \in \Rpd$ such that for all
  $\x \in \D$,
  %
  \begin{eqnarray*}
    &c_{1} \nx \leq \V\arx \leq c_{2} \nx,\\
    &\dV\arx + c_{3} \V\arx \leq 0.
  \end{eqnarray*}
\end{definition}

From the above definition, it is easy to see that
\begin{align*}
  \frac{d\V\arx}{dt} + c_{3} \V\arx \leq 0 \Rightarrow \V(t) \leq \V_{0} \, e^{c_{3} t},
\end{align*}
where $\V_{0} = \V(\x\argbzero)$.

A natural extension of Lyapunov functions is the notion of control Lyapunov
functions which generalize the time derivative to include a control term.
%
For a system with no control, the time derivative of a function $\V : \Rnn \to
\R$ can be expressed as
\begin{align*}
  \Lie{\xf} \V\arx &= \frac{d\V\arx}{dt}
\end{align*}
where $\Lie{\xf} \V\arx$ represents the Lie derivative (see, e.g.,
\cite{Sastry1999}) of the function $\V\arx$ along the vector field
$\xf$, i.e.,
\begin{align*}
  \Lie{\xf} \V\arx = \pd{\V\arx}{\x} \cdot \frac{d\x}{dt}.
\end{align*}
%
If control acts on the system as in
\begin{align}
  \label{eq:ccsys-es}
  \dx = \xf + \xg \, \uu,
\end{align}
the time derivative becomes
\begin{align*}
  \dx = \Lie{\xf} \V\arx + \Lie{\xg} \V\arx \, \uu.
\end{align*}
%
This understanding of differentiation along vectors fields motivates the
introduction of control Lyapunov functions:
%
\begin{definition}
  \label{def:res-clf}
  For a system of the form \eqref{eq:ccsys-es}, a continuously differentiable
  function $\V : \D \to \Rnn$ is said to be a {\bf \em control Lyapunov function
    (CLF)} if there exist constants $c_{1}, c_{2}, c_{3}, c_{4} \in \Rpd$ such
  that for all $\x \in \D$,
  %
  \begin{eqnarray*}
    &c_{1} \nx \leq \V\arx \leq c_{2} \nx,\\
    &\inf_{\uu \in \U} \left[ \Lie{\xf} \V\arx + \Lie{\xg} \V\arx \, \uu + c_{3} \V\arx \right] \leq 0.
  \end{eqnarray*}
\end{definition}

% 
In order to guarantee a stricter form of convergence, a modified notion of these
functions is used (c.f. \cite{Ames2014}):

\begin{definition}
  \label{def:res-clf}
  For the continuous dynamics of a system of the form \eqref{eq:hsys}, a continuously differentiable function $V_{\resclfparam} : \D \to \Rnn$ is said to be a {\bf \em rapidly exponentially stablizing control Lyapunov function (RES--CLF)} (see \cite{Ames2014}) if there exist constants $c_{1}, c_{2}, c_{3}, c_{4} \in \Rpd$ such that for all $\resclfparam > 0$ and for all $\x \in \D$,
  %
  \begin{eqnarray*}
    &c_{1} \nx \leq \Ve(\x) \leq \frac{c_{2}}{\resclfparam^{2}} \nx,\\
    &\inf_{\uu \in \U} \left[ \Lie{\bigF} \Ve(\x) + \Lie{\bigG} \Ve(\x) \, \uu + \frac{c_{3}}{\resclfparam} \Ve(\x) \right] \leq 0.
  \end{eqnarray*}
\end{definition}

For the continuous dynamics, define a candidate Lyapunov function, $\V : \D \to \Rnn$, of the form
%
\begin{align}
  \label{eq:lyap}
  \Ve(\x) = \frac{1}{2} (\E(\x) - \Eref)^2,
\end{align}
%
with $\Eref$ the energy associated with the periodic orbit, and use it as a control Lyapunov function to construct the energy shaping controller
%
\begin{align}
  \label{eq:es-qp}
  \mueps(\x) = \argmin_{u\arx \in \R^{m}} \ & \uu\arx^T \uu\arx\\
  \nonumber
  \mbox{s.t. } \Lie{\xf} \Ve&\arx + \Lie{\xg} \Ve\arx \, \uu\arx + \frac{c_{3}}{\resclfparam} \Ve\arx \leq 0.
\end{align}
%
Applying this to the system \eqref{eq:hcsys} results in
%
\begin{align}
  \label{eq:hsys-cl}
  \HS = \left\{
  \begin{array}{l l}
    \dx = \xf\arx + \xg\arx \, \mueps\arx, & \xm \in \D \setminus \Guard,\\
    \xp = \Delta\arxm, &\xm \in \Guard.
  \end{array}\right.
\end{align}
%
As a result of the control law construction in \eqref{eq:es-qp}, the closed-loop
dynamics of \eqref{eq:hsys-cl} is stabilized with respect to the zero level set
of the Lyapunov function \eqref{eq:lyap} thus satisfying the convergence
guaranteed specified in \defref{def:res-clf}.

As an added remark, it bears mentioning that a closed-form continuously
differentiable solution to the QP \eqref{eq:es-qp} is given by the min-norm
controller (see, e.g., \cite{Freeman1996}), 
\begin{align}
  \label{eq:min-norm}
  \mueps\arx = -\frac{\psieps_{0}\arx \psieps_{1}\arx}{(\psieps_{1}\arx)^{T} \psieps_{1}\arx},
\end{align}
where
\begin{align*}
  \psieps_{0}\arx := \Lie{\xf} \Ve\arx + \frac{c_{3}}{\resclfparam}, \quad
  \psieps_{1}\arx := \left( \Lie{\xg} \Ve\arx \right)^{T}.
\end{align*}
This solution is practically useful for several reasons.
%
Most importantly, it provides a closed-form solution which is important for
proving \thmref{theorem:main-theorem} (which is stated subsequently in \secref{sec:stab}).
%
In addition, it can be used in numerical simulations in place of a quadratic
program to reduce the computational overhead (in most cases).
%
In the simulations described in this work in \chapref{chap:simulations}, the
min-norm solution \eqref{eq:min-norm} is used to reduce computation time.

  


\section{Stability of the Shaped System} \label{sec:stab}

With the preceding setup in mind, the main formal idea behind energy shaping can now be stated:\vgap
%
\begin{theorem}
  \label{theorem:main-theorem}
  Given an exponentially-stable limit cycle in a hybrid system of the form \eqref{eq:hsys}, application of the energy shaping controller \eqref{eq:es-qp} to the control system \eqref{eq:hcsys} results in the closed-loop hybrid system \eqref{eq:hsys-cl}, which is exponentially stable.\vgap
\end{theorem}
%

The proof is given later after some discussion.


In order to achieve the stated goal, it is necessary to show that, given a system with a limit cycle representing the desired behavior, energy shaping can be applied and the resulting system will have an invariant orbit which is equivalent to the nominal system. Simply put, the control contribution from the energy shaping controller must be identically zero on the orbit. Consider the following lemma:\vgap

\begin{lemma}
  Applying the energy shaping controller \eqref{eq:es-qp} to a hybrid control system \eqref{eq:hcsys} results in a closed-loop hybrid system \eqref{eq:hsys-cl} that demonstrates a periodic orbit which is identical to the unshaped system \eqref{eq:hsys}.\vgap
\end{lemma}

\begin{proof}
  For states on the periodic orbit, i.e., $\xo \in \orbit$, the energy is a known constant, $E(\xo) = \Eref$.
  %
  Therefore, the limit cycle represents an invariant level set of the energy.
  %
  By construction of the Lyapunov function \eqref{eq:lyap} used in the controller \eqref{eq:es-qp}, it is clear that $V(\xo) = 0$ and, moreover, that
  %
  \begin{align*}
    \inf_{\x \in \D} V(\x) = 0.
  \end{align*}
  %
  The solution to the optimization problem \eqref{eq:es-qp} has cost $u(\x)^T u(\x) = 0$ (which implies that all elements of $u(\x)$ are zero when $V(\x) = 0$) and this satisfies the stability condition of the control Lyapunov function;
  %
  indeed ${\dot V}(\xst) = 0$ since the energy does not change without external forcing.
  %
  Thus, the periodic orbits are equivalent.
\end{proof}

\section{Zero Dynamics Formulation} \label{sec:zd-form}

In order to understand the nature of energy shaping, consider breaking up the system into two sets of coordinates,
%
\begin{align}
  \nonumber
  \dot \zdx &= f\argsxz + g\argsxz \, u,\\
  \dot \zdz &= q\argsxz + w\argsxz \, u,
  \label{eq:zd-vfield}
\end{align}
%
with states $\zdx \in \zdX$ and $\zdz \in \zdZ$ and control inputs $\uu \in \U$.
%
The vector fields $f$, $g$, $q$, and $w$ are assumed to be locally Lipschitz continuous.
%
To simplify notation, define
%
\begin{align*}
  \bigF\argsxz = \left(\begin{array}{c}
    f\argsxz\\
    q\argsxz
  \end{array}\right), \quad
  \bigG \argsxz = \left(\begin{array}{c}
    g\argsxz\\
    w\argsxz
  \end{array}\right).
\end{align*}
%
The natural choice of transformation to convert the continuous dynamics of \eqref{eq:hcsys} to \eqref{eq:zd-vfield} is through energy.
%
Thus, for mechanical systems where $\x = (\q, \dq) \in T\Q$, consider the coordinate transformation
%
\begin{align*}
  \zdx &= \E(\x) - \Eref,\\
  \zdz &= \xrem,
\end{align*}
where $n$ is the size of the configuration space, $\Q$.
%
By construction, the fixed point of the hybrid system can be chosen to occur at $\argsxz = \xzst$ such that $\Delta\xzst = \argszero$.
%
Moreover, because energy does not change by the natural dynamics, $f\argsxz = 0$.
%

%
The coordinate transformation is valid if it is locally diffeomorphic.
%
For mechanical systems, which are of interest to this paper, energy takes the form
%
\begin{align}
  \label{eq:mech-sys-nrg}
  E(\q, \dq) = \frac{1}{2} \dq^{T} \M(\q) \dq + U(\q),
\end{align}
%
where the first and second terms represent the kinetic and mechanical energy, respectively.
%
Because $\M(\q)$ is invertible for mechanical systems, it never has determinant zero.
%
In addition, the Jacobian matrix associated with the transformation is
\begin{align*}
  \pd{\Psi(\q, \dq)}{(\q, \dq)} =
  \left(\begin{array}{c c}
    I & \boldzero_{2n-1 \times 1}\\
    \pd{E(\q, \dq)}{\xrem} & \pd{E(\q, \dq)}{\dq_{n}}
  \end{array}\right).
\end{align*}
%
This matrix is invertible when
\begin{align*}
  \det \Psi\argsxz = \pd{E(\q, \dq)}{\dq_{n}} \neq 0.
\end{align*}
From inspection of \eqref{eq:mech-sys-nrg}, one can see that the coordinate transformation is almost everywhere diffeomorphic for physical systems.

\section{Exponential Stability}

Applying the control law \eqref{eq:es-qp}, the dynamics \eqref{eq:zd-vfield} becomes
%
\begin{align}
  \nonumber
  \dot \zdx &= f\argsxz + g\argsxz \, \mueps\argsxz,\\
  \dot \zdz &= q\argsxz + w\argsxz \, \mueps\argsxz.
  \label{eq:zd-vfield-cl}
\end{align}

By the construction of the control law \eqref{eq:es-qp}, it is clear that $\mueps(0, \zdz) = 0$ and thus it follows that $f(0, \zdz) = 0$.
%
In other words, the zero dynamics manifold $\zdmanifold$ is the restricted subset of $X$ such that $\zdx = 0$.
%
Rewrite the candidate Lyapunov function \eqref{eq:lyap} in the zero dynamics coordinates,
%
\begin{align*}
  V(\zdx) = \frac{1}{2} \zdx^{2}
\end{align*}
and consider the following:\vgap

\begin{proposition}  
  \label{prop:res-clf}
  Exponential stability of the continuous $\zdx$ dynamics is guaranteed if a RES--CLF exists satisfying
  %
  \begin{eqnarray}
    \label{eq:lyap-cond-nrg-es}
    &c_{1} \nzdx^{2} \leq \V\argsx) \leq \frac{c_{2}}{\resclfparam^{2}} \nzdx^2,\\
    \nonumber
    &\inf_{\uu \in \U} \left[ \Lie{\bigF} \V\argsxz + \Lie{\bigG} \V\argsxz \, \uu + \frac{c_{3}}{\resclfparam} \V\argsx \right] \leq 0,
  \end{eqnarray}
  %
  for all $\argsxz \in \zdX \times \zdZ$.\vgap
\end{proposition}


\begin{proof}
  It is easy to see that the first inequality is satisfied for $c_{1} \leq \frac{1}{2}$ and $c_{2} \geq \frac{\resclfparam^{2}}{2}$.
  %
  Define the set
  %
  \begin{align}
    \Keps = \left\{ \uu \in \U : \Lie{\bigF} \V\argsxz + \Lie{\bigG} \V\argsxz \, \uu + \frac{c_{3}}{\resclfparam} \V(\zdx) \leq 0 \right\}.
    \label{eq:control-set}
  \end{align}
  %
  In order to see that the set $\Keps$ is not empty, consider the control Lypaunov function system in the original coordinates as given in \eqref{eq:lyap}.
  %
  The time derivative is
  %
  \begin{align*}
    \dVe(\x) &= \E(\x) \pd{\E(\x)}{\x}(f(\x) + g(\x) \, \uu)\\
    &\geq \E(\q, \dq) \left(\!\!\begin{array}{c c}
    \pd{\E(\q, \dq)}{\q} & \pd{\E(\q, \dq)}{\dq}
    \end{array}\!\!\right) \cdot
    \left[
      \left(\!\!\begin{array}{c}
      \dq\\
      -\M^{-1}(\q) \CG(\q, \dq)
      \end{array}\!\!\right) +
      \left(\!\!\begin{array}{c}
      \boldzero_{\ndim \times \ncv}\\
      -\M^{-1}(\q) \B(\q)
      \end{array}\!\!\right) \, \uu
      \right].
  \end{align*}
  %
  For a fully-actuated system, $\ncv = \ndim$ so $\B : \Q \to \R^{\ndim \times \ncv}$ is full rank.
  %
  Since $M(\q)$ is invertible for physical bodies, $M(\q) B(\q)$ is invertible.
  %
  Thus there exists at least one $\uu(\q, \dq)$ satisfying
  %
  \begin{align*}
    \M^{-1}(\q) \B(\q) \, \uu(\q, \dq) = \frac{c_{3}}{\resclfparam} \frac{1}{2}
    \E^{2}(\q, \dq) + \pd{\E(\q, \dq)}{\q} \dq - \pd{\E(\q, \dq)}{\dq}
    \M^{-1}(\q) \CG(\q, \dq).
  \end{align*}
  %
  and, for the CLF \eqref{eq:lyap-cond-nrg-es} with any such locally Lipschitz continuous feedback control law $u : \zdX \times \zdZ \to \K$, it follows that solutions satisfy
  %
  \begin{align}
    \label{eq:xt-bounds}
    \| \zdx(t) \| \leq \frac{1}{\resclfparam} \sqrt{\frac{c_{2}}{c_{1}}} e^{-\frac{c_{3}}{2\resclfparam} t} \| \zdx(0) \|
  \end{align}
  for all $t \geq 0.$
\end{proof}

A hybrid system can be constructed to describe the system under the zero
dynamics coordinates by combining the continuous dynamics \eqref{eq:zd-vfield}
with the reset map (expressed in the zero dynamics coordinates) viz.
%
\begin{align}
  \label{eq:zd-hcsys}
  \HCSbar= \left\{
    \begin{array}{l c l}
      \begin{array}{r c l}
        {\dot \zdx} &=& f\argsxz + g\argsxz \, \uu\\
        {\dot \zdz} &=& q\argsxz + w\argsxz \, \uu
      \end{array} &
      \left.\vphantom{
          \begin{array}{r c l}
            {\dot \zdx} &=& f\argsxz + g\argsxz \, \uu\\
            {\dot \zdz} &=& q\argsxz + w\argsxz \, \uu
          \end{array}}\right\} & \mbox{ if } \argsxz \in \D \setminus \Guard,\\
      % 
      \begin{array}{r c l}
        \zdx^{+} &=& \DeltaX(\zdx^{-}, \zdz^{-})\\
        \zdz^{+} &=& \DeltaZ(\zdx^{-}, \zdz^{-})
      \end{array}
      & \left.\vphantom{\begin{array}{r c l}
            \zdx^{+} &=& \DeltaX(\zdx^{-}, \zdz^{-})\\
            \zdz^{+} &=& \DeltaZ(\zdx^{-}, \zdz^{-})
          \end{array}}\right\}
      & \mbox{ if } \argsxz \in \Guard.
    \end{array}\right.
\end{align}
% 
Applying a valid Lipschitz continuous control law (which takes values in \eqref{eq:control-set}) to the hybrid control system \eqref{eq:zd-hcsys} results in the closed loop hybrid system
%
\begin{align}
  \label{eq:zd-hsys-cl}
  \HSbar_{\resclfparam} = \left\{\!\!\!\!
  \begin{array}{l c l}
    \begin{array}{r c l}
      {\dot \zdx} &=& f\argsxz + g\argsxz \, \mueps\argsxz\\
      {\dot \zdz} &=& q\argsxz + w\argsxz \, \mueps\argsxz
    \end{array} &
    \left.\vphantom{\begin{array}{r c l}
          {\dot \zdx} &=& f\argsxz + g\argsxz \, \mueps\argsxz\\
          {\dot \zdz} &=& q\argsxz + w\argsxz \, \mueps\argsxz
        \end{array}}\right\} & \mbox{ if } \argsxz \in \D \setminus \Guard,\\
    % 
    \begin{array}{r c l}
      \zdx^{+} &=& \DeltaX(\zdx^{-}, \zdz^{-})\\
      \zdz^{+} &=& \DeltaZ(\zdx^{-}, \zdz^{-})
    \end{array}  &
    \left.\vphantom{\begin{array}{r c l}
          \zdx^{+} &=& \DeltaX(\zdx^{-}, \zdz^{-})\\
          \zdz^{+} &=& \DeltaZ(\zdx^{-}, \zdz^{-})
        \end{array}}\right\} & \mbox{ if } \argsxz \in \Guard.
  \end{array}\right.
\end{align}

\section{Proof of Main Result} \label{sec:proof}

Let the \Poincare{} map of \eqref{eq:hsys-cl} be denoted $\Pe : \Guard \to \Guard$ and let $\flow_{t}\argsxz$ represent a flow of the vector field for time $t$ starting from state $\argsxz$.
%
The \Poincare{} map takes the form
%
\begin{align}
  \label{eq:poincare}
  \Pe\argsxz = \floweps_{\TIe\argsxz}(\Delta\argsxz),
\end{align}
%
where $\TIe\argsxz$ is the time to impact.
%
Before proving the main theorem, some bounds related to the \Poincare{} map must be established using arguments similar to those presented in \cite{Ames2014,Morris2005}.

First, however, note that since the reset map is locally Lipschitz continuous about the fixed point $\argsxz = \xzst$ and because $\DeltaX\xzst = 0$, it follows that
%
\begin{align} 
  \nonumber
  \left\| \DeltaX\argsxz - \DeltaX\xzst \right\| &\leq \left\| \Delta\argsxz - \Delta\xzst \right\|
  \\
  \label{eq:impact-Lipschitz}
  &\leq \LDelta \nzdxzst
\end{align}
%
for some $\argsxz \in \B_{\gamma}\xzst$ with $\gamma > 0$ where $\LDelta$ is the Lipschitz constant of $\Delta\argsxz$.

Now, consider the following bounds on the time-to-impact functions and \Poincare{} maps:\vgap

\begin{lemma}
  \label{lemma:TIe-P-bounds}
  For the control system \eqref{eq:zd-hsys-cl},
  %
  \begin{eqnarray*}
    &| \TIe(\Deltaxz) - \TI(\Deltaxz) | \leq \ATIe \nzdxzst,\\
    &\| \Pe\argsxz - \P\argsxz \| \leq \Ae \nzdxzst,
  \end{eqnarray*}
  %
  where $\limeps \ATIe = 0$ and $\limeps\Ae = 0$.\vgap
\end{lemma}
\begin{proof}
  Consider the \Poincare{} section $\PSection$ which is the guard from \eqref{eq:guard}.
  %
  Using the change of coordinates $\rho(\resclfparam) := \tfrac{1}{\resclfparam}$, define the function
  %
  \begin{align*}
    N(t, \rho, \zdx, \zdz) = h(\flowrhot(\Deltaxz)),
  \end{align*}
  %
  which is locally Lipschitz continuous in $\zdx$, $\zdz$, and $\rho$ by construction as a composition of Lipschitz continuous functions.
  %
  By the transversality assumption, it follows that
  %
  \begin{align*}
    \pd{N(T, 0, 0, \zdzst)}{t} &= \dot h(\flow^{0}_{t} (\Delta(0, \zdzst)) \ne 0.
  \end{align*}

  %
  Thus, by the implicit function theorem (\cite{Sun2001}), there exists a $\delta > 0$ and a unique function $\taurho(\rho, \zdx, \zdz)$ defined and locally Lipschitz for all $(\rho, \zdx, \zdz) \in \B_{\delta}(0, 0, \zdzst)$ such that $\tau^{0}(0, \zdzst) = \TI(0, \zdzst) = \Tst$ where $\Tst$ is the period of the invariant orbit $\Orbit$ and
  %
  \begin{align*}
    N(\taurho\argsxz, \rho, \zdx, \zdz) = 0.
  \end{align*}
  %
  In addition, selecting a fixed  $\resclfparam > 0$ and a Lipschitz continuous feedback control law $\mueps \in \Keps\argsxz$ such as the min-norm controller \cite{Freeman1996}, it follows that for some (possibly smaller) $\delta > 0$ and $\argsxz \in \B_{\delta}\xzst \cap \Guard$, the time-to-impact function $\TIe\argsxz$ satisfies
  %
  \begin{align}
    \label{eq:TIe-bounds}
    0.9 \Tst \leq \TIe\argsxz \leq 1.1 \Tst.
  \end{align}
  %
  
  The rest of the proof involves constructing an auxiliary time-to-impact function, $\TB$, which is locally Lipshitz continuous and independent of $\resclfparam$, and then relating it to $\TIe$.
  %
  
  Let $\eta \in \R^{n}$ be a constant vector and, using the guard $h\argsxz$, define
  %
  \begin{align*}
    \TB(\eta, \zdx, \zdz) = \inf\{t \geq 0 : h(\eta + \flowt(\Delta(0, \zdzst))) = 0\}.
  \end{align*}
  %
  From this, it directly follows that $\TB(0, \zdx, \zdz) = \TI\argsxz$.
  %
  Just as with $\TIe$, $\TB$ is locally Lipschitz continuous.
  %
  Thus,
  %
  \begin{align*}
    | \TB(\eta, \zdx, \zdz) - \TI\argsxz | \leq \LB \| \eta \|
  \end{align*}
  %
  where $\LB$ is the local Lipschitz constant.

  Let $\argsxzti{1}$ and $\argsxzti{2}$ satisfy
  %
  \begin{align*}
    \argsDxzti{1} &= \floweps\argsxzti{1},\\
    \argsDxzti{2} &= \flow\argsxzti{2},
  \end{align*}
  %
  respectively, with initial conditions
  %
  \begin{align*}
    \argsxzizero{1} = \argsxzizero{2} = \Delta(0, \zdzst).
  \end{align*}
  %
  Define
  %
  \begin{align}
    \label{eq:etaeps}
    \etaeps = \left.\argsxzti{1}\right|_{t = \TIe\argsxz} - \left.\argsxzti{2}\right|_{t = \TIe\argsxz}
  \end{align}
  %
  and as a result,
  %
  \begin{align*}
    \TIe\argsxz = \TB(\etaeps, \zdx, \zdz).
  \end{align*}
  %
  Now $\mu$ must be bounded.

  The explicit solution to the QP \eqref{eq:es-qp} is given by the min-norm control law \cite{Freeman1996}:
  %
  \begin{align*}
    \mueps\argsxz = -\frac{\psieps_{0}\argsxz \psieps_{1}\argsxz}{(\psieps_{1})^{T}\argsxz \psieps_{1}\argsxz},
  \end{align*}
  %
  with
  %
  \begin{align*}
    \psieps_{1}\argsxz := \Lie{\bigF} \Ve\argsxz + \frac{c_{3}}{\resclfparam}, \quad
    \psieps_{2}\argsxz := \left( \Lie{\bigG} \Ve\argsxz \right)^{T},
  \end{align*}
  %
  where $\Lie{\bigF} \Ve\argsxz \equiv 0$ by the choice of $\Ve\argsxz$ as energy.
  %
  Since energy does not change by the natural dynamics,
  %
  \begin{align*}
    \mueps\argsxz = - \frac{\frac{c_{3}}{\resclfparam} \Ve\argsxz
      \bigG^{T}\argsxz \left( \pd{V_{e}\argsxz}{\argsxz}
      \right)^{T}}{\pd{V_{e}\argsxz}{\argsxz} \bigG\argsxz \bigG^{T}\argsxz
      \left( \pd{V_{e}\argsxz}{\argsxz} \right)^{T}}.
  \end{align*}
  %
  If, in addition, $\bigG$ is full rank and takes values in a compact set for states near the orbit, i.e., $\bigG\argsxz \in \stabletube$ and $d_{1} \leq \| \bigG\argsxz \| \leq d_{2}$, then
  %
  \begin{align}
    \nonumber
    \| \mueps\argsxz \|
    &\leq \frac{\frac{c_{2}c_{3}}{\resclfparam^3} \nzdx^{2} \left\| \bigG^{T}\argsxz
      \left(\!\begin{array}{c}
      \zdx\\
      \boldzero
      \end{array}\!\right)\right\|}{\left\|
      \left(\!\begin{array}{c c}
      \zdx & \boldzero
      \end{array}\!\right) \bigG\argsxz \bigG^{T}\argsxz
      \left(\!\begin{array}{c}
      \zdx\\
      \boldzero
      \end{array}\!\right) \right\|}\\
    \nonumber
    &\leq \frac{\frac{c_{2}c_{3}}{\resclfparam^{3}} \nzdx^{2} \left\| \bigG\argsxz
      \, \zdx \right\|}{\left\|
      \sum_{i} \bigG_{i,i}^{2}\argsxz \right\| \nzdx^{2}}\\
    \label{eq:mueps-bounds}
    &\leq \frac{c_{2}c_{3}}{\resclfparam^{3}} \frac{\lambdamaxG}{\lambdaminG} \nzdx,
  \end{align}
  %
  with
  %
  \begin{align*}
    \lambdamaxG &:= \sup \left\{\lambdamax\bigG\argsxz : \argsxz \in \stabletube \right\},\\
    \lambdaminG &:= \inf \left\{\lambdamin\bigG\argsxz : \argsxz \in \stabletube \right\},
  \end{align*}
  %
  where $\stabletube \subset \D$ is a stable tube around the periodic orbit $\Orbit$.
  %
  In addition, by the Lipschitz assumption $\bigF$ is bounded around $\Orbit$, hence
  %
  \begin{align*}
    \supF := \sup \left\{ \left\| \bigF\argsxz \right\| : \argsxz \in \stabletube \right\}.
  \end{align*}

  The expression $\etaeps$ in \eqref{eq:etaeps} is bounded using \eqref{eq:mueps-bounds} and \eqref{eq:xt-bounds}:
  %
  \begin{align}
    \nonumber
    \|\etaeps\| &\leq \int_{0}^{\TIe\argsxz} \! \left\| \bigG\argsxztau \, \mueps\argsxztau \, \right\| d\tau\\
    \nonumber
    &\leq \frac{c_{2}c_{3}}{\resclfparam^{3}} \frac{\lambdamaxG^{2}}{\lambdaminG^{2}} \int_{0}^{\TIeDelta} \! \nzdxt d\tau\\
    \nonumber
    &\leq \frac{c_{2}^{\frac{3}{2}}c_{3}}{c_{1}^{\frac{1}{2}}\resclfparam^{4}} \frac{\lambdamaxG^{2}}{\lambdaminG^{2}} \nzdxzero \int_{0}^{\TIeDelta} \! \left|e^{-\frac{c_{3}}{2\resclfparam} \tau} \right| d\tau\\
    \nonumber
    &\leq \frac{c_{2}^{\frac{3}{2}} c_{3}^{2}}{2c_{1}^{\frac{1}{2}}\resclfparam^{5}} \frac{\lambdamaxG^{2}}{\lambdaminG^{2}} \nzdxzero \, \left. e^{-\frac{c_{3}}{2\resclfparam} \tau} \right|^{0}_{\tau=\TIeDelta}.\\
    \nonumber
    &\leq\frac{c_{2}^{\frac{3}{2}} c_{3}^{2}}{2c_{1}^{\frac{1}{2}}\resclfparam^{5}} \frac{\lambdamaxG^{2}}{\lambdaminG^{2}}  \, \left(1 - e^{-\frac{c_{3}}{2\resclfparam} 1.1 \Tst}\right) \left\|\DeltaX\argsxz - \DeltaX\xzst\right\|\\
    \label{eq:G-bound}
    &\leq \LDelta \frac{c_{2}^{\frac{3}{2}} c_{3}^{2}}{2c_{1}^{\frac{1}{2}}\resclfparam^{5}} \frac{\lambdamaxG^{2}}{\lambdaminG^{2}}  \, \left(1 - e^{-\frac{c_{3}}{2\resclfparam} 1.1 \Tst}\right) \nzdxzst.
  \end{align}
  %
  Using this bound with $\TB$ leads to
  %
  \begin{align*}
    | \TIe\argsxz &- \TI\argsxz |\\
    &= | \TB(\etaeps, \zdx, \zdz) - \TB(0, \zdx, \zdz)|\\
    &\leq \LB \| \etaeps \|\\
    &\leq \LB \LDelta \frac{c_{2}^{\frac{3}{2}}
      c_{3}^{2}}{2c_{1}^{\frac{1}{2}}\resclfparam^{5}}
    \frac{\lambdamaxG^{2}}{\lambdaminG^{2}}  \, \left(1 -
      e^{-\frac{c_{3}}{2\resclfparam} 1.1 \Tst}\right) \nzdxzst.
  \end{align*}
  %
  Defining $\ATIe$ as the coefficient of $\nzdxzst$ in the preceding equation establishes the first part of the lemma.

  To complete the proof, consider the bound
  %
  \begin{align}
    \label{eq:flow-bound-comp}
    \lefteqn{\left\| \Pe\argsxz - \P\argsxz \right\| \leq}\\
    \nonumber
    && \int_{\TIDelta}^{\TIeDelta} \left\| \bigF\argsxztau \right\| \, d\tau  + \int_{0}^{\TIeDelta} \left\| \bigG\argsxztau \, \mueps\argsxztau \, \right\| d\tau.
  \end{align}
  %
  For simplicity, consider the two terms separately.
  %
  The first term can be bounded as follows:
  %
  \begin{align}
    \nonumber
    \int_{\TIDelta}^{\TIeDelta)} \! \left\| \bigF\argsxztau \right\| \, d\tau
    &\leq \supF \int_{\TIDelta}^{\TIeDelta} d\tau\\
    \nonumber
    &\leq \left.\supF \tau\right|_{\tau=\TIDelta}^{\TIeDelta}\\
    \nonumber
    &\leq \supF \left\|\TIeDelta - \TIDelta\right\|\\
    &\leq \supF \ATIe \nzdxzst.
    \label{eq:F-bound}
  \end{align}
  %
  The second term has already been bounded in \eqref{eq:G-bound}.
  %
  Finally, combining \eqref{eq:G-bound}, \eqref{eq:flow-bound-comp}, and \eqref{eq:F-bound} results in
  %
  \begin{align*}
    \left\| \Pe\argsxz - \P\argsxz \right\| \leq \Ae \! \nzdxzst,
  \end{align*}
  %
  where
  %
  \begin{align*}
    \Ae := \LDelta \frac{c_{2}^{\frac{3}{2}}
      c_{3}^{2}}{2c_{1}^{\frac{1}{2}}\resclfparam^{5}}
    \frac{\lambdamaxG^{2}}{\lambdaminG^{2}}  \, \left(1 -
      e^{-\frac{c_{3}}{2\resclfparam} 1.1 \Tst}\right) +
    \ATIe\nzdxzst. %&\qedhere
  \end{align*}
  %
  From this expression, it is clear that the limiting behavior is $\limeps \Ae = 0$.
\end{proof}


Now \thmref{theorem:main-theorem} can be proven:

\begin{proof}{\em [\thmref{theorem:main-theorem}]}
  By the discrete converse Lyapunov theorem, exponential stability of $\Orbit$ implies the existence of a discrete Lyapunov function $\Vn : B_{\delta}\xzst \cap \Guard \to \Rnn$ satisfying
  %
  \begin{align}
    \label{eq:lyap-cond-nom}
    r_{1} \nzdxz^{2} \leq \Vn\argsxz &\leq r_{2} \nzdxzst^2,\\
    \nonumber
    \Vn(\P\argsxz) - \Vn\argsxz &\leq -r_{3} \nzdxzst^{2},\\
    \nonumber
    | \Vn\argsxz - \Vn(\zdx', \zdz') | &\leq
    r_{4} \| \argsxzst - \argsxzstprime \| \cdot\\
    \nonumber
    & \hspace{3em} (\nzdxzst + \nzdxzstprime)
  \end{align}
  %
  for some $r_{1}, r_{2}, r_{3}, r_{4} \in \Rpd$.
  %
  In addition, consider the CLF associated with \eqref{eq:es-qp} which is $\Ve : \zdX \to \Rnn$.
  %
  Denote by $\Vex = \VeS$ the restriction of the CLF $\Ve$ to the switching surface $\Guard$.
  %
  Using these Lyapunov functions, define the candidate Lyapunov function
  %
  \begin{align}
    \VP\argsxz = \Vn\argsxz + \sigma \Vex(\zdx).
  \end{align}
  %
  From \eqref{eq:lyap-cond-nrg-es} and \eqref{eq:lyap-cond-nom}, it is apparent that $\VP\argsxz$ is bounded as follows:
  %
  \begin{align*}
    \sigma c_{1} \nzdx^{2} + r_{1} \nzdxzst^{2} \leq \VP\argsxz \leq \sigma
    \frac{c_{2}}{\resclfparam^{2}} \nzdx^{2} + r_{2} \nzdxzst^{2}.
  \end{align*}
  %
  Next, note that
  %
  \begin{align}
    \label{eq:composite-lyap}
    \lefteqn{\VP(\Pe\argsxz) - \VP\argsxz =}\\
    \nonumber
    &&\Vn(\Pe\argsxz) - \Vn\argsxz + \sigma(\Vex(\Pe\argsxz) - \Vex(\zdx)).
  \end{align}
  %
  By construction of the control law \eqref{eq:es-qp}, it is true that
  %
  \begin{align}
    \label{eq:Veh-bound}
    \Vex(\zdx) &\leq \frac{c_{2}}{\resclfparam^{2}} \left\| \zdx \right\|^{2},\\
    \nonumber
    \Vex(\Pe^{\zdx}\argsxz) &\leq \frac{c_{2}}{\resclfparam^{2}} e^{-\frac{c_{3}}{\resclfparam} \TIe\argsxz} \left\| \DeltaX\argsxz \right\|^{2}.
  \end{align}
  %
  Combining \eqref{eq:impact-Lipschitz} and \eqref{eq:Veh-bound} yields
  %
  \begin{align*}
    \Vex(\Pe^{\zdx}\argsxz) &\leq \frac{c_{2}}{\resclfparam^{2}} e^{-\frac{c_{3}}{\resclfparam} \TIe\argsxz} \left\| \DeltaX\argsxz - \DeltaX\xzst \right\|^{2}\\
    &\leq \frac{c_{2}}{\resclfparam^{2}} e^{-\frac{c_{3}}{\resclfparam} \TIe\argsxz} \LDelta^{2} \nzdxzst^{2}.
  \end{align*}
  %
  Using \lemref{lemma:TIe-P-bounds} and defining $\bepsone := \frac{c_{2}}{\resclfparam^{2}} \LDelta^{2} e^{-\frac{c_{3}}{\resclfparam} .9 \Tst}$ yields
  %
  \begin{align}
    \label{eq:DVeX-bounds}
    \Vex(\Pex\argsxz) - \Vex(\zdx) \leq \bepsone \nzdxzst^{2} - \frac{c_{2}}{\resclfparam^{2}} \nzdx^{2}.
  \end{align}

  Now the Lyapunov function from the converse theorem must be bounded.
  %
  As a result of \lemref{lemma:TIe-P-bounds} and the assumption of exponential stability about the origin, it follows that
  %
  \begin{align}
    \label{eq:Pe-P-bounds}
    &\| \Pe\argsxz - \P\argsxz \| \leq \Ae \nzdxzst,\\
    \nonumber
    &\| \Pe\argsxz \| = \| \Pez\argsxz - \P\argsxz + \P\argsxz - \P\xzst\|\\
    \nonumber
    & \hspace{4.4em} \leq \Ae \nzdxzst  + \LP\nzdxzst,\\
    \nonumber
    &\| \P\argsxz \| \leq N \alpha \nzdxzst,
  \end{align}
  %
  where $\LP$ is the Lipschitz constant for $\P$.
  %
  Then, using \eqref{eq:lyap-cond-nom},
  %
  \begin{align*}
    \lefteqn{| \Vn(\Pe\argsxz) - \Vn(\P\argsxz) | \leq}\\
    &&r_{4} \| \Pe\argsxz - \P\argsxz \| \left( \|\Pe\argsxz\| + \|\P\argsxz\| \right).
  \end{align*}
  %
  Applying \eqref{eq:Pe-P-bounds} to the above results in
  %
  \begin{align}
    \label{eq:Pe-P-beta}
    | \Vn(\Pe\argsxz) - \Vn(\P\argsxz) | \leq \bepstwo \nzdxzst^{2}
  \end{align}
  %
  where, for simplicity, $\bepstwo := r_{4} \Ae (N \alpha + \Ae  + \LP)$.
  %
  Application of \eqref{eq:Pe-P-beta} to the nominal part of \eqref{eq:composite-lyap} gives the following bounds:
  %
  \begin{align}
    \label{eq:nom-lyap-bounds}
    \lefteqn{\Vn(\Pe\argsxz) - \Vn\argsxz }\\
    \nonumber
    &= \Vn(\Pe\argsxz) - \Vn(\P\argsxz) + \Vn(\P\argsxz) - \Vn\argsxz\\
    \nonumber
    &\leq \bepstwo \nzdxzst^{2} - r_{3} \nzdxzst^{2}.
  \end{align}
  %
  Combining \eqref{eq:composite-lyap}, \eqref{eq:DVeX-bounds}, and \eqref{eq:nom-lyap-bounds} yields
  %
  \begin{align*}
    \VP(\Pe\argsxz) - \VP\argsxz \leq - \left(\begin{array}{c c}
      \nzdx & \nzdzst
    \end{array}\right)
    \Le
    \left(\begin{array}{c}
      \nzdx\\
      \nzdzst
    \end{array}\right)
  \end{align*}
  %
  with
  %
  \begin{align*}
    \Le = \left(\begin{array}{c c}
      r_{3} - \bepstwo - \sigma (\bepsone - \frac{c_{2}}{\resclfparam^{2}}) &
      r_{3} - \bepstwo - \sigma \bepsone\\
      r_{3} - \bepstwo - \sigma \bepsone & r_{3} - \bepstwo - \sigma \bepsone
    \end{array}\right).
  \end{align*}
  %
  The matrix $\Le$ is positive definite if
  %
  \begin{align*}
    \det \Le = \sigma\frac{c_{2}}{\resclfparam^{2}} (r_{3} - \bepstwo - \sigma \bepsone) > 0,
  \end{align*}
  %
  and thus stability is achieved when
  %
  \begin{align*}
    &\bepstwo + \sigma \bepsone < r_{3}.
  \end{align*}
  %
  Examining the limits, it becomes apparent that
  %
  \begin{align*}
    \limeps \bepsone = 0, \qquad \limeps \bepstwo = 0,
  \end{align*}
  %
  and thus for small enough values of $\sigma > 0$ and large enough values of $\resclfparam$, stability is maintained.
\end{proof}
