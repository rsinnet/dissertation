\chapter{\uppercase{Introduction}}

This thesis discusses energy shaping as a means for improving the stability
properties of periodic orbits in mechanical systems.
%
Energy shaping is applicable to a wide range of mechanical systems but this work
is chiefly concerned with continuous systems that have intermittent impacts
\cite{Brogliato1996} which result in non-smooth solutions.
%
Such a description (not unintentionally) categorizes bipedal walking which is
currently an area of widespread interest and is the focal point of the
application of energy shaping in this thesis.
%
By developing energy shaping under the framework of hybrid systems, dealing with
robotic locomotion becomes straightforward.
%

The method presented is similar in concept to the method of total energy shaping
as presented in \cite{Spong2007}, which acts to shape the energy of the system
but does so in a way which only guarantees asymptotic stability with respect to
an arbitrary energy level and does not guarantee exponential stability of the
overall gait and may in fact destabilize gaits.
%
The method of energy shaping presented herein, in contrast, can improve the
stability properties of a hybrid periodic orbit while guaranteeing that
stability is not lost.
%


Numerous methods currently exist for gait design but, aside from specific
methods which construct a zero dynamics such as human-inspired control
\cite{Grizzle2014}, many of these methods do not have an intrinsic concept of
stabilization to a specific gait through gain adjustment.
%
This is especially true of passivity-based methods such as \csx \cite{Spong2005}
and other controlled Lagrangian methods \cite{Bloch2001,Bloch2000}.
%
Energy shaping owes its development to the observation that the total energy of
a system is conserved in the absence of non-conservative forcing.
%
For systems which do exhibit non-conservative forcing, the energy added or
removed can be tracked using a storage functions and the sum of the total energy
of the system and the storage function is conserved;
%
systems which demonstrate this property are called passivity-based systems
\cite{Spong2007}.
%

When such a conserved energy quantity exists, it is, by definition, constant for
periodic orbits of a system.
%
Thus for a hybrid system which exhibits discontinuities, this quantity is
constant through the continuous dynamics but experiences jumps due to the
discrete dynamics.
%
As a result, it seems reasonable to conclude that stability is largely a result
of the discrete dynamics.
%
An easily digestible example of this phenomenon can be observed in an
uncontrolled compass-gait biped which, as McGeer observed \cite{McGeer1990}, is
capable of walking stably down shallow slopes given the appropriate model
parameters;
%
this example is presented in \secref{sec:simulations-compass-gait} and a
historical context is provided in \secref{sec:literature-passive-walkers}.

For hybrid systems---systems which combine continuous dynamics such as leg
swing with discrete dynamics such as foot-strike---the conservation of energy
through the continuous dynamics means that the change in energy level occurs
from the discrete events in the system---foot-strike for bipeds---which exert
non-conservative impulsive forcing as a byproduct of an interaction with the
ground.
%
By adding control to the continuous dynamics, overall stability properties of a
gait tend to improve as has been observed in, e.g., \cite{Spong2003}, and as will
be demonstrated later in this paper through simulation in
\chapref{chap:simulations}.
%
Formulation of the control objective using a control Lyapunov function makes it
possible to achieve these improvements while simultaneously guaranteeing the
existence of a control law which does not destabilize the system.

It is important to clarify the specific advantages conferred by energy shaping.
%
One advantage of the method which is presented in this thesis is a formal
guarantee of stability.
%
However, the stability which is proven is local exponential stability and
nothing is formally shown about stability at arbitrary distances from the valid
operating region; \secref{sec:hsys-stability} discusses stability in more detail
and information about historical context can be found in
\secref{sec:literature-stability}.

In addition to the formal results, the claim is also made (as has been made
before \cite{Spong2003}) that energy shaping can improve the robustness of
gaits.
%
Different notions of robustness exist throughout the literature and one that is
rather well known is robustness with respect to model uncertainties
(cf. \cite{Freeman1996}).
%
The theory presented does not treat this form of robustness and assumes perfect
knowledge of the model.
%
Another definition which crops up is robustness with respect to perturbations in
initial conditions which ties into domain of attraction \cite{Chesi2011}.
%
This particular definition is the one considered in this thesis and numerical
simulations seem to substantiate the claim that energy shaping improves
robustness with respect to initial conditions thereby increasing the domain of
attraction.
%
It may be possible for specific systems, by guessing valid Lyapunov functions,
to make formal statements about the domain of attraction, but such investigation
is beyond the scope of this work.

In addition to energy shaping, this thesis contains a chapter on human-inspired
control, which is a framework that was created to achieve human-like walking in
bipedal robots.
%
By examining human kinematics data, it becomes apparent that certain
kinematics outputs of human walking, termed human outputs, can be encoded with
very little loss by fitting the data to canonical walking functions, which have
the same form as the solution to a linear spring--mass--damper system.
%
In addition, the methods of hybrid zero dynamics \cite{Morris2005} are drawn
upon to construct a zero dynamics which can be rendered forward-invariant
through feedback linearization, and is, moreover, invariant through impacts
(i.e., foot strike), resulting in guaranteed stability properties
\cite{Ames2012}.
%
Simulations are given to demonstrate human-inspired control in
\chapref{ch:hic}.