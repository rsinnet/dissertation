\chapter{\uppercase{Literature Overview}}

This chapter describes the literature which pertains to bipedal robotic walking,
reviewing work from different fields and perspectives such as biomechanics and
dynamical systems.
%
Bipedal robotic walking is inherently interdisciplinary including theoretical
and experimental components and drawing inspiration from locomotion found in
nature.
%
The exposition is divided by field:
%
research on the stability of dynamical systems is discussed first followed by
modeling of hybrid systems.
%
The chapter ends with a review of bipedal robots and control methods from the
literature.

\section{Stability of Dynamical Systems} \label{sec:literature-stability}

Stability of dynamical systems has a long history, appearing in the works of
researchers such as Routh \cite{Routh1877} and Hurwitz \cite{Hurwitz1895}.
%
Contemporary notions of stability in nonlinear systems generally rely on results
initially presented in the work of Aleksandr Lyapunov in his doctoral thesis in
1892 \cite{Lyapunov1992}.
%
In said treatise, Lyapunov described two methods for analyzing the stability of
equilibrium points of ordinary differential equations.

The {\em first method}, sometimes called the {\em indirect method of Lyapunov},
provides a means for understanding the stability properties of an equilibrium
point by examining a linearization of the nonlinear system.
%
Due to the nature of linearization, the results of such an analysis only pertain
locally and within an unknown region about the equilibrium point.
%
Nonetheless, this method is well-known and sees widespread use due to its
simplicity and straightforward nature.
%

The {\em second method}, sometimes called the {\em direct method of Lyapunov}
involves the use of scalar-valued functions of the state of a system called {\em
  Lyapunov functions} which satisfy specific conditions that are set forth
later.
%
Through the use of these Lyapunov functions, it is possible to prove
stability not only locally but in a known region containing an equilibrium point
or even globally.
%
These functions are not unique and a major drawback to this method is the lack
of an algorithmic procedure for constructing valid Lyapunov functions.

Both methods are used in this work but for different purposes:
%
Lyapunov's indirect method is often used to analyze the stability of hybrid
systems by examining the stability of a linearization of the \Poincare{} map
about the equilibrium point; this will be explained in greater detail in
\secref{sec:hsys-stability}.
%
The numerical simulations in this work will rely on this usage of Lyapunov's
indirect method.
%
In order to formally demonstrate the stability of energy shaping---the main
focus of this work---Lyapunov's direct method will be employed.
%
This method is often used in theoretical constructions and can also be used to
understand domain of attraction, although the particular usage will preclude
this type of application.

Lyapunov's work on the direct method established sufficient conditions for
stability but lacked the notion of {\em uniform stability} which was required
for establishing necessary conditions.
%
The results presented by Lyapunov lay essentially dormant for decades until
researchers began to investigate the ideas further.
%
From 1930's onward, researchers established additional results expanding on
Lyapunov's ideas.
%
Khalikoff \cite{Khalikoff1937} and Malkin \cite{Malkin1938} proved additional
theorems on stability which could be used to show stability in the sense of
Lyapunov with relaxed assumptions.
%
As explained in \cite{Murakami1988}, using Lyapunov's direct method, Marachkoff
\cite{Marachkoff1940} provided a proof of asymptotic stability on systems of the
form $\dx = f(t, x)$ using a negative definite Lyapunov function;
%
the negative definiteness requirement was later relaxed to negative semidefinite
through the addition of a second Lyapunov function by Matorosov
\cite{Matorosov1962}.
%

%
Masera \cite{Massera1949} provided more restrictive definitions, introducing the
notion of {\em equiasymptotic stability}.
%
Yet it wasn't until the assumptions of uniformity were provided by Malkin
\cite{Malkin1954} that the necessary framework existed in which to formulate
converse theorems.
%
Barbashin and Krasovski\u{\i} further strengthened Malkin's result in
\cite{Barbashin1954}.
%

After these observations were published, converse theorems followed shortly
thanks to researchers such as Kurzweil \cite{Kurzweil1956} and Massera
\cite{Massera1956}.
%
Converse theorems have seen substantial development and broad use since these
results.
%
Hoppensteadt \cite{Hoppensteadt1966} presented constructions for singularly
perturbed systems in which he constructed a $C^{1}$ Lyapunov function, extending
existing work on singularly perturbed systems to unbounded time intervals.
%
Wilson \cite{Wilson1969} constructed a $C^{\infty}$ Lyapunov function for a
continuous vector field having an asymptotically stable invariant set.
%
Early results on converse Lyapunov functions for stability of sets are
summarized in numerous texts; see, e.g, \cite{Antosiewicz1958}.

In addition to providing necessary and sufficient conditions for stability,
researchers have also studied what conditions are necessary for solutions to be
integrable for all time.
%
Cesari provides conditions for \cite[\S 1.5]{Cesari1971} for second-order linear
systems.
%
Strauss introduces $L^{p}$ stability to attempt to provide conditions for more
general systems in \cite{Strauss1965}.

As LaSalle points out \cite{LaSalle1964}, before the 1960's much of the work
done in the USSR was inaccessible to English-speaking audiences, but as the
decade progressed, this language barrier gradually dissipated.
%
An early text by LaSalle and Lefschetz (the first such text in English)
\cite{LaSalle1961} outlining the methods of Lyapunov stability theory contains
proofs which are accessible to those with less extensive mathematical
backgrounds.
%
Shortly thereafter, additional texts emerged including those of Krasovski\u{\i}
\cite{Krasovskii1963} and of Hahn \cite{Hahn1967}.
%
Yoshizawa's text from 1975 \cite{Yoshizawa1975} provides a compilation of
results on Lyapunov stability in almost periodic systems.

Additional information on the history of Lyapunov theory can be found throughout
the literature; see, e.g., \cite{Michel2007, Teel1999}.
%

\section{Hybrid Systems}

The term hybrid systems derives its name from the mixed nature of the dynamics
involved.
%
In general, systems which combine continuous dynamics with discrete dynamics are
referred to as such.
%
Contemporary notions of hybrid systems evolved over the past few decades;
%
the term hybrid systems itself is broadly and loosely used in different areas of
control.
%
Because of the combination of dynamics, hybrid systems often contain solutions
with discontinuities as discussed by Filippov \cite{Filippov1988}.

Many of the developments in hybrid systems have been motivated by applications:
%
G{\"{o}}ll{\"{u}} and Varaiya \cite{Gollu1989} applied hybrid dynamical systems to
formulate a control scheme for hard disk drives.
%
In a later work \cite{Varaiya1993}, Varaiya also described automated
vehicle--highway systems.
%
Similarly, Tomlin, Pappas and Sastry \cite{Tomlin1998} described automated
automated air traffic management systems.
%
Brockett \cite{Brockett1993} described hybrid models for motion control systems,
attempting to provide generalizations for a wide class of systems.
%
Such multilayer control schemes as those described earlier by, e.g., Brooks
\cite{Brooks1986}, in some sense provided an impetus for more formal
descriptions.
%
Simpler examples include thermostat/furnace systems and surge tanks as described
by, e.g., Antsaklis, Stiver and Lemmon \cite{Antsaklis1993}, and the even
simpler cat and mouse game mentioned by Manna and Pnueli \cite{Manna1993}.

Harel \cite{Harel1987} presented the idea of statecharts as a visual formalism
for modeling of hybrid systems.
%
Other researchers have proposed various languages for hybrid systems such as
Hooman \cite{Hooman1993} and Benveniste, Borgne and Guernic \cite{Benveniste1993}.




\section{Bipedal Robotic Locomotion} \label{sec:literature-bipeds}

\begin{figure}
  \centering
  \def\svgwidth{1.0\columnwidth}
  \input{figs/biped-models-2.eps_latex}
  \caption[Four low-dimensional models]{%
    Low-dimensional models are often as approximations of walking robots.
    %
    From left to right:
    %
    Linear Inverted Pendulum, Inverted Pendulum with Flywheel, Spring-Loaded
    Inverted Pendulum, Compass-Gait Biped
    % 
    the Linear Inverted Pendulum Model (LIPM) assumes a lumped center-of-mass
    with constant height and massless legs;
    %
    the Inverted Pendulum with Flywheel (IPF) adds a flywheel at the CoM to
    model angular momentum and allows for varying height;
    %
    the Spring-Loaded Inverted Pendulum (SLIP) considers a spring to model the
    legs as massless pogo sticks;
    %
    and the Compass-Gait Biped (CG) is essentially a double pendulum with
    lumped masses on the stance and swing legs and one at the hip.}
  \label{fig:biped-models}
\end{figure}

Bipedal robotics has been approaches in numerous different ways from the
analysis of passive walkers based on simple mechanical design principles to
advanced multifunctional humanoids.
%
Control designs have been proposed based on models which vary from the
low-dimensional representations shown in \figref{fig:biped-models} to
higher-dimensional dynamic models as developed in \chapref{ch:modeling}.
%
While simpler models offer the benefits of faster control law computation and
easier-to-understand analyses, more complex models can provide tighter control
and formal guarantees on stability.
%
In this section, some well-known approaches to bipedal gait generation
are discussed emphasizing the effect of modeling on control design.
%
Further information is available in \cite{Chevallereau2009, Full1999,
  Holmes2006, Hurmuzlu2004, Kuo2007, Sadati2012, Siciliano2008, Westervelt2007,
  Wisse2007} and the references therein.


\subsection{Zero Moment Point and Linear Inverted Pendulum Models}

The ZMP control strategy \cite{Vukobratovic2004, Vukobratovic1990} is extremely
common in control of bipeds.
%
The ZMP is the point on the ground at which the reaction forces acting between
the ground and the foot produce no horizontal moment.
%
Traditionally, ZMP control strategies achieve walking by planning the motion of
a robot's CoM such that the ZMP remains strictly within the convex hull of the
stance foot in the case of single support (or convex hull of the stance feet, in
the case of double support).
%
Under this condition on the ZMP, the stance foot remains flat on the ground and
immobile (not rotating)---much like the base of a traditional manipulator
robot---and hence the robot will not topple; see, e.g., \cite{Yamaguchi1999}.

In the special case of the Linear Inverted Pendulum Model (LIPM),
%
%When the height of the CoM is constant as in the LIPM,
%
the ZMP can be treated with an ordinary differential equation modeling the
center-of-mass (CoM) dynamics.
%
This treatment assumes representation of a robot as a point mass with massless telescoping legs.
%
Moreover, the height of the CoM is assumed to be constant.
%
Under these conditions, \cite{Kajita1991} proved that a model reduces to the LIPM.
%
%The ZMP control strategy uses the linear ODE to plan the motion of a robot's
%CoM such that the ZMP remains strictly within the convex hull of the stance
%foot in the case of single support (or convex hull of the stance feet, in the
%case of double support).

The LIPM assumes the height of the center of mass (CoM) and angular momentum
about the stance foot are constant through a step.
%
Because they share some key assumptions, the ZMP control method and the LIPM
have historically been coupled.
%
While the LIPM originated in the study of human posture and balance (e.g.,
\cite{Geursen1976, Winter1995, Patton1999}), it has also been the focus of much
research in bipedal locomotion; see, for example, \cite{Miura1984, Kajita2001,
  Kajita2010}.
%
%This simplified model is often used with control methods that generate walking
%gaits by controlling the ZMP.



Early experimental work on bipedal robots came from Japan, where Kato began
building the WABOT series of humanoid robots circa 1970.
%
A full-scale anthropomorphic robot, WABOT-1, was described in \cite{Kato1974} and
it was capable of primitive, statically stable walking while carry objects with
its hands.
%
Years later, the ZMP technique was first demonstrated in experiment on the
WL10-RD biped in \cite{Takanishi1985}.
%
The study of walking humanoid robots has increased in pace with researchers from
designing newer generations of robots like WABIAN-2 \cite{Ogura2006}, ASIMO
\cite{Sakagami2002}, HRP-4 \cite{Kaneko2011}, KHR-3 \cite{Park2005}, and Johnnie
\cite{Pfeiffer2002}.n


Though ZMP methods have been vastly successful, a number of drawbacks still
presents challenges.
%
By the nature of the models, ZMP gaits usually do not consider impacts and so
the swing foot trajectory must be specified such that ground impact is minimal.
%
Moreover, ZMP condition alone is not sufficient for asymptotic stability of
periodic walking \cite{Choi2005}.
%
Despite this, researchers have pursued numerous methods of gait generation using
ZMP:
%
Nagasaka et al \cite{Nagasaka1999} used the optimal gradient method;
%
Kajita et al \cite{Kajita1992} examined potential energy conserving orbits;
%
Lim et al \cite{Lim2002} computed ZMP-consistent trajectories off-line and
stabilized them using trunk motion;
%
Nishiwaki et al \cite{Nishiwaki2002} generated ZMP-consistent trajectories
in real-time while walking;
%
and Kurazume et al \cite{Kurazume2003} used analytical solutions to the ZMP
dynamics;
%
Additional information on ZMP-based methods and related ground reference points
is given in \cite{Goswami1999, Vukobratovic2004, Vukobratovic2006,
  Popovic2005}.

%Other contributions include \cite{Yamaguchi1999}.
%BHR-2 \cite{Jarfi2006}, SURALP \cite{Taskiran2010}.    china and turkey!!!
%also, REEM-B from Spain
%what about COMAN from Italy (no walking reported in publication?)?
%

\subsection{Nonlinear Inverted Pendulum Models}

In order to overcome limitations resulting from the simplicity of the LIPM
model, researchers have considered more complex models.
%
Park et al \cite{Park1998} explored the Gravity Compensated LIPM which adds an
additional point mass at the location of the swing foot to achieve higher
modeling accuracy.
%
In \cite{Pratt2007}, Pratt and Drakunov relaxed the requirement of constant CoM
height on the LIPM leading to a nonlinear inverted pendulum model.
%
In another common model, a flywheel is added to the inverted pendulum;
%
examples can be found throughout the literature:
%
Stephens \cite{Stephens2011} used it for posture control,
%
Takenaka et al \cite{Takenaka2009} used it to with on-line error compensation to
mitigate the effect of modeling errors on gait generation,
%
and Komura et al \cite{Komura2005} used it to simulate pathological gaits.
%
The various pendulum models have been widely used in analysis of push recovery
and balance \cite{Takanishi1990, Hof2005, Hyon2007, Stephens2007}.

%However, the capture point method is often used with simply the LIPM model (\cite{Englsberger2012}) to achieve walking.

%over the past decade, researchers have taken further forays into the feedback
%control of humanoid robots, examining aspects such as balance and push recovery
%\cite{PCDG2006, Stephens_humanoidpush2007, Pratt2012, Koolen2012}.
%
Pratt et al \cite{Pratt2006} considered a flywheel model in order to present the
idea of the {\em capture point}---a point on the ground on which a biped can
step and come to a complete (upright) stop without falling over;
%
additional information on capture points can be found in \cite{Koolen2012,
  Pratt2012}.
%
The capture point \cite{Pratt2006a} has gained recognition as a convenient
method for stabilizing a biped.
%
Intuitively, a capture point is a point on the ground where a biped can place
its swing foot to be able to avoid falling over---the set of all capture points
is called the {\em capture region}.
%
This method, which considers a robot as an inverted pendulum with a flywheel,
has been used not only for standing but for robust walking as well.
%
Because the model makes many simplifying assumptions, the capture regions can
have a large error and this has motivated the combining of capture point with
learning in, e.g., \cite{Rebula2007}.

\subsection{Spring-Loaded Inverted Pendulum Models}

In the 1980's, a large amount of pioneering work was done at MIT by Marc
Raibert.
%
Raibert constructed a spring-loaded leg and tested new ideas using
low-dimensional models for hopping and running
%
His two-dimensional hopper could ``run'' at a speed of 1 m/s
\cite{Raibert1984, Raibert1986, Raibert1984a}.
%
A three-dimensional version of the origin hopper was also successful
\cite[Chap.~3]{Raibert1986} followed up by multi-legged robots
\cite{Hodgins1991, Raibert1990, Raibert1986a}.
%
This work led the way for researchers studying the Spring-Loaded Inverted
Pendulum (SLIP) model, which has been shown to approximate the center-of-mass
dynamics of \textit{steady-state running gaits} on a numerous animals from
insects all the way to humans \cite{Blickhan1989, Mcmahon1990, Farley1993,
  Full2000, Dickinson2000, Seyfarth2002}.
%
Using Raibert's principles, other simple robots have achieved running such as
the ARL-Monopod II \cite{Ahmadi2006} and the CMU Bowleg Hopper
\cite{Zeglin1998}.
%

%The control of these robots has been based on Raibert's original control ideas,
%which can be decomposed into three subtasks dedicated to (a) forward propulsion
%of the robot at the desired speed, (b) regulation of the vertical hopping height
%of the body, and (c) keeping the body at a desired posture (\cite{Raibert1984},
%\cite[Ch. 2]{Raibert1986}).
%%
%To control the forward speed, the controller places the toe at a desired
%position with respect to the center of mass during flight.
%%
%To regulate the hopping height, the length of the leg at the bottom of the
%stance phase is adjusted by giving a fixed amount of thrust.
%%
%Finally, to control the pitch attitude of the body, the controller employs hip
%torque during stance.
%%
More complex robots were later build with knees and compliance including the
Spring Flamingo and Spring Turkey \cite{Hollerbach1992, Pratt1999, Pratt2000,
  Pratt2001}.
%
These robots used a type of series elastic actuator (SEA) designed for force
control as opposed to energy storage.
%
The recent COMAN robot discussed in \cite{Li2013} includes passive
compliance to reduce energy consumption during walking.


\subsection{Passive Walkers and the Compass-Gait Biped} \label{sec:literature-passive-walkers}

In contrast to the minimalist models just discussed, some researchers create
robots with dynamics designed to realize a simplified model and thus, in such
cases, mechanical design plays an important role in dynamic stability.
%
Even these complex designs, however, often achieve walking through controllers
designed for low-dimensional models such as the ZMP strategy.
%
Low-dimensional models are generally easier to model and it is more straightforward
to take into account impact dynamics.
%
As impacts play an important role in dynamic stability, this confers a valuable
advantage over gaits designed with simpler methods such as LIPM methods.
%

In 1980, Mochon and McMahon \cite{Mochon1980} pointed out that human walking is
similar to a double pendulum thereby hinting at the passive nature of human
walking and the importance of mechanical design.
%
In the late 80's, McGeer built planar, passive bipedal walkers which helped to
validate the claims of Mochon and McMahon.
%
His bipeds which included a compass-gait walker \cite{McGeer1990} and kneed version \cite{McGeer1990a} could walk stably down shallow slopes with no actuation.
%
This began a trend of studying {\em passive dynamic walking}.
%
Based on McGeer's original ideas, researchers have constructed minimalist bipeds
which achieve walking by injecting and removing small amounts of energy as described in \cite{Collins2005}
%
The result is ``human-looking'' walking, but the remarkable elegance and
economy of these walkers comes at the cost of poor ability in achieving tasks
other than walking at a fixed speed;
%
they cannot climb stairs, pause, turn or run.
%

More in-depth work was later done to analyze the properties of passive walkers,
for instance, in \cite{Espiau1994, Garcia1998, Borzova2004}.
%
Spong \cite{Spong1999} looked at passive dynamic walking with energy-based
methods to design passivity-based control strategies such as controlled
symmetries, introduced in \cite{Spong2005}, which was later used to obtain
anthropomorphic foot action in \cite{Sinnet2009, Sinnet2009a} and recently
extended to underactuated bipeds in \cite{Hu2011}.
%
Spong's work on energy shaping methods such as controlled symmetries provides a
starting point for the methods provided in this thesis.
Other important contributions to passive dynamic walking are given in
\cite{Anderson2005, Kuo1999, Kuo2002, Wisse2007}.


\subsection{Quadratic Programs and Lyapunov Funnels}

Early implementations of ZMP-based controllers used off-line trajectory
optimization to generate center of mass trajectories on the basis of the LIPM
and generally ignored impact dynamics in the control design.
%
Modern methods have achieved improved control by formulating the control problem
as a {\em quadratic  program} (QP), allowing gait replanning on the fly and
improving stability properties \cite{Kudoh2002, Stephens2010, Herdt2010}.
%

Similarly, sums of squares methods, also formulate trajectory generation as a
convex optimization \cite{Tedrake2010}.
%
One elegant method which provides formal guarantees on stability is outlined in
\cite{Majumdar2013}.
%
The procedure investigates the notion of controllability, composing sequential
funnels (verified with sum of squares Lyapunov inequalities), which each lead to
a predefined goal set.
%
This allows one to create a trajectory with guaranteed stability properties:
%
at any given point in time, the system is within one of the known funnels
(regions of attraction).
%
By using the sum of squares formulation, the trajectory optimization becomes
more tractable, making verification of stability feasible for low-dimensional
models.
