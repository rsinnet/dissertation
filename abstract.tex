%%%%%%%%%%%%%%%%%%%%%%%%%%%%%%%%%%%%%%%%%%%%%%%%%%%
%
%  New template code for TAMU Theses and Dissertations starting Fall 2012.  
%  For more info about this template or the 
%  TAMU LaTeX User's Group, see http://www.howdy.me/.
%
%  Author: Wendy Lynn Turner 
%	 Version 1.0 
%  Last updated 8/5/2012
%
%%%%%%%%%%%%%%%%%%%%%%%%%%%%%%%%%%%%%%%%%%%%%%%%%%%
%%%%%%%%%%%%%%%%%%%%%%%%%%%%%%%%%%%%%%%%%%%%%%%%%%%%%%%%%%%%%%%%%%%%%
%%                           ABSTRACT 
%%%%%%%%%%%%%%%%%%%%%%%%%%%%%%%%%%%%%%%%%%%%%%%%%%%%%%%%%%%%%%%%%%%%%

\chapter*{ABSTRACT}
\addcontentsline{toc}{chapter}{ABSTRACT} % Needs to be set to part, so the TOC doesnt add 'CHAPTER ' prefix in the TOC.

\pagestyle{plain} % No headers, just page numbers
\pagenumbering{roman} % Roman numerals
\setcounter{page}{2}

\indent This thesis presents a method which attempts to improve the stability
properties of periodic orbits in hybrid dynamical systems by shaping the
energy.
%
By taking advantage of conservation of energy and the existence of invariant
level sets of a conserved quantity of energy corresponding to periodic orbits,
energy shaping drives a system to a desired level set.
%
This energy shaping method is similar to existing methods but improves upon them
by utilizing control Lyapunov functions, allowing for formal results on
stability.
%
The main theoretical result, \thmref{theorem:main-theorem}, states that, given
an exponentially-stable limit cycle in a hybrid dynamical system, application of
the presented energy shaping controller results in a closed-loop system which is
exponentially stable.

The method can be applied to a wide class of problems including bipedal
locomotion;
%
because the optimization problem can be formulated as a quadratic program
operating on a convex set, existing methods can be used to rapidly obtain
the optimal solution.
%
As illustrated through numerical simulations, this method turns out to be useful
in practice, taking an existing behavior which corresponds to a periodic orbit
of a hybrid system, such as steady state locomotion, and providing an
improvement in convergence properties and robustness with respect to
perturbations in initial conditions without destabilizing the behavior.
%
The method is even shown to work on complex multi-domain hybrid systems;
%
an example is provided of bipedal locomotion for a robot with non-trivial foot
contact which results in a multi-phase gait.

\pagebreak{}
