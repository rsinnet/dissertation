\documentclass{beamer}

%\usetheme{PaloAlto}
\usetheme{Szeged}
%\usecolortheme{beaver}

\title[Energy Shaping]{A Lyapunov Approach to Orbital \\Stabilization through Energy Shaping}
\subtitle{Preliminary Results}
\author{R. W. Sinnet}
\institute{Department of Mechanical Engineering\\ Texas A\&M University}
\date{July ?, 2014}


\begin{document}

\frame{\titlepage}

\begin{frame}
  \frametitle{Table of Contents}
  \tableofcontents
\end{frame}

\section{Introduction}
\frame[intro]{intro}

\section{Energy in Mechanical Systems}
\frame[Mechanical Systems]{mech}

\frame[Conservative Systems]{csys}

\frame[Nonconservative Systems]{ncsys}

\section{Orbital Stabilization}
\begin{frame}
  \begin{itemize}
    \item Periodic orbits have associated energy levels which define a hypersurface.
    \item We can stabilize to an energy level.
  \end{itemize}
\end{frame}

\begin{frame}
  \frametitle{A Simple Example}
  The pendulum can be stabilized to a given level set of the energy.
\end{frame}

\frame[Pendulum]{Pendulum}

\section{Hybrid Systems}
\begin{frame}
  \frametitle{Hybrid Systems}
  hsys
\end{frame}

\begin{frame}
  \frametitle{Simple Hybrid Behavior}
  \begin{itemize}
    \item Compass-Gait Biped
    \item Three-Link Biped
  \end{itemize}
\end{frame}

\frame{Complex Hybrid Behavior}


\section{Walking with Feet}
\begin{frame}
  \frametitle{Walking with Feet}
  \begin{itemize}
  \item 2D biped with feet
  \item 3D biped with feet
  \end{itemize}
\end{frame}

\section{Conclusions}
\frame[Conclusions]{con.}

%\section{Hybrid Systems}
%
%\subsection{Formalisms}
%\frame{hsys}
%
%\section{Hybrid Models}
%\frame{hmodels}
%
%\section{Controlled Lagrangians}
%\begin{frame}
%  \begin{itemize}
%    \item Explain the significance of energy in mechanical systems.
%  \end{itemize}
%\end{frame}

%\section{Energy Shaping}
%\frame{ES}
%\frame[Formulation]{s}

\end{document}
