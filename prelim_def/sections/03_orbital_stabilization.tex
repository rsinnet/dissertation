\section{Orbital Stabilization}
\showtoc

\subsection{Orbital Stabilization with Control Lyapunov Functions}
\begin{frame}
  \frametitle{Motivation}
  \begin{block}{Main Question}
    Can we use an understanding of energy exchange to improve global stability properties of periodic orbits in mechanical systems?
  \end{block}

  \begin{block}{Observations}
    \begin{itemize}
    \item Numerous control design schemes exist for stabilizing mechanical systems to periodic orbits.
    \item Some controllers produce good behavior locally but lack robustness.
    \item Periodic orbits have associated energy functions with level sets which are invariant under the orbits.
    \end{itemize}
  \end{block}
\end{frame}

\begin{frame}[t]
  \frametitle{Overview}
  \only<1>{
    \begin{block}{Setup}
      Consider the control system 
      \begin{align*}
        \dot x = f(x) + g(x) u(x).
      \end{align*}
      Assume there exists a control law ${\bar u}(x)$ which creates a limit cycle in the closed-loop dynamics,
      \begin{align*}
        {\bar f}(x) = f(\q, \dq) + g(\q) {\bar u}(\q, \dq).
      \end{align*}
      Also assume there exists an energy function $E_{c} : T\ConfigurationSpace \to \R$ which is conserved, i.e., $E_{c}(\q, \dq) \equiv E_{0}$, on the limit cycle.
    \end{block}
  }

  \only<2>{
    \begin{block}{Main Idea}
      Add robustness to a periodic behavior by imposing convergence on an energy function to a level set which is known to be invariant under the system dynamics.
    \end{block}
    
    \begin{block}{Control Objective}
      Choose control input $\mu(\q, \dq)$ such that $\| \mu(\q, \dq) - {\bar u}(\q, \dq) \|$ is minimized and $E_{c}(\q(t), \dq(t)) \to E_0$ as $t \to \infty$.
    \end{block}

    \begin{block}{Exponential Convergence}
      To achieve exponential stabilization, $E_{c}(x(t))$ should satifisy
      \begin{align*}
        E_{c}(\q(t), \dq(t)) \leq E_{c}(\q(t_{0}), \dq(t_{0})) e^{-\beta t} \mbox{ for } t \geq t_{0}, \beta > 0.
      \end{align*}
    \end{block}
  }
\end{frame}

\begin{frame}
  \frametitle{Control Lyapunov Functions}
  A \blue{control Lyapunov function} $V : X \to \R$ which satisfies
  \begin{align*}
    &c_{1} \| x \|^2 \leq V(x) \leq c_{2} \|x\|^2,\\
    &\inf_{u \in U} L_f V(x) + L_g V(x) \, u + c_{3} V(x) \leq 0
  \end{align*}
  for $c_{1}, c_{2}, c_{3} > 0$ exhibits exponential convergence.
\end{frame}

\begin{frame}
  \frametitle{Energy Shaping}
  Consider a conserved energy function $E_{c}(x)$. For an exponentially stablilzing CLF, we seek $\mu(x)$ such that
  \begin{align*}
    L_f V(x) + L_g V(x) \, \mu(x) + \epsilon V(x) &\leq 0.
  \end{align*}
  Choosing $V(x) = \eta^2$ with $\eta(x) = E_{c}(x) - E_{0}$, we obtain
  \begin{align*}
    2 \eta(x) \left(L_f \eta(x) + L_g \eta(x) \, \mu(x) \right) + \epsilon \eta^2(x) \leq 0.
  \end{align*}
  We can relax this condition by augmenting the optimization space with $\delta \in \R$ and requiring
  \begin{align*}
    2 \eta(x) \left(L_f \eta(x) + L_g \eta(x) \, \mu(x) \right) + \epsilon \eta^2(x) \leq \delta(x).
  \end{align*}
\end{frame}

\begin{frame}[t]
  \frametitle{Quadratic Program Formulation}
  The linear form of the CLF condition suggests
  \begin{align}
    \nonumber
    (\delta^*(x), \mu(x)) = \argmin_{v = (\delta, u)}  \, & v^T \! H \, v + h^T(x) v\\
    \label{clf} \tag{clf}
    \mbox{s.t. } & \Aclf(x) \, v \leq \bclf(x)%\\
    %\label{lim} \tag{lim}
    %& \Alim v \leq \blim
  \end{align}
  where \eqref{clf} imposes the control Lyapunv function. To encode the dynamics of the system under control input ${\bar u}(x)$, select
  \begin{align*}
    H = \left(\begin{array}{c c}\nu & 0\\ 0 & I\end{array}\right), & \quad
      h(x) = \left(\begin{array}{c} 0\\ -2 \, \bar u(x) \end{array}\right),\\
      \Aclf = \left(\begin{array}{c c}
        -1 & 2 \eta(x) \, L_{g} \eta(x)
      \end{array}\right), & \quad
      \bclf = -\epsilon \eta^{2}(x) - 2\eta(x) \, L_{f} \eta(x).
  \end{align*}
  with positive weighting coefficient $\nu \in \R$.
\end{frame}

\subsection{Types of Systems}
\begin{frame}
  \frametitle{Conservative Systems}
  For hybrid systems with non-dissipative continuous dynamics, the total energy functions as the necessary conserved energy function,
  \begin{align*}
    E_{c}(x) = T(x) + U(x).
  \end{align*}
  This ties in with the method of \blue{Controlled Lagrangians}, increasing the robustness of Lagrangian shaping.
\end{frame}

\begin{frame}
  \frametitle{Example: Compass Gait as a Shaped System}
  \only<1>{
    \begin{columns}
      \column{1.5in}
      Dynamic Model:
      \begin{align*}
        M(q) \ddot q + H(q, \dot q) = B(q) u
      \end{align*}
      Control Law:
      \begin{align*}
        \bar u(q) &= G(q) - G(\Psi(q))
      \end{align*}

      \column{1.5in}
      \begin{figure}
        \centering
        \def\svgwidth{1.0\columnwidth}
        \input{figures/cg2d-2link-model.eps_latex}
        \vspace{-2em}
        \caption{Compass-gait biped with Controlled Symmetries.}
      \end{figure}
    \end{columns}
  }
  \only<2>{
    \begin{figure}
      \centering
      \includegraphics[width=1.0\columnwidth]{energy_cg2d_slope_model}
      \caption{Energy of the shaped system is conserved.}
    \end{figure}    
  }
  %  \only<3>{
  %    \includemedia[
  %      width=1.0\columnwidth,
  %      height=0.5625\columnwidth,
  %      addresource=amber2d.mp4,
  %      activate=pageopen,
  %      flashvars={source=amber2d.mp4&loop=true&autoPlay=true}
  %    ]{}{}%VPlayer9.swf}
  %  }

  \only<3>{
    Energy shaping can be achieved using:
    \begin{align}
      \nonumber
      \argmin_{v = (\delta, u)}  \, & v^T \! H \, v + h^T(q, \dot q) v\\
      \label{clf} \tag{clf}
      \mbox{s.t. } & \Aclf(q, \dot q) v \leq \bclf(q, \dot q)
    \end{align}
    where
    \begin{align*}
      H = \left(\begin{array}{c c}\epsilon & 0\\ 0 & I\end{array}\right), \qquad
        h(q, \dot q) = \left(\begin{array}{c} 0\\ -2 \, \bar u(q, \dot q) \end{array}\right),
    \end{align*}
    and
    \begin{align*}
      \Aclf = \left(\begin{array}{c c}
        -1 & 2 \eta L_{g} \eta
      \end{array}\right), \qquad
      \bclf = -\epsilon \eta^{2} - 2\eta L_{f} \eta
    \end{align*}
  }

  \only<4>{
    \begin{figure}
      \centering
      \includegraphics[width=1.0\columnwidth]{es_comparison_2link_conservative}
      \caption{Demonstration of energy shaping on 2-link biped.}
    \end{figure}
  }

  \only<5>{
    \begin{figure}
      \centering
      \includemedia[
        %width=1.0\columnwidth,
        %height=0.5625\columnwidth,
        width=1.0\columnwidth,
        height=0.5\columnwidth,
        addresource=cg2d_es.mp4,
        activate=pageopen,
        flashvars={source=cg2d_es.mp4&loop=true&autoPlay=true}
      ]{}{VPlayer9.swf}
      \caption{Energy shaping to stabilize to a gait from distant initial condition.}
    \end{figure}
  }
\end{frame}


\begin{frame}
  \frametitle{Nonconservative Systems}
  For hybrid systems with nonconservative continuous dynamics, the energy exchange must be considered, leading to the conserved energy function
  \begin{align*}
    E_{c}(\q, \dq) = T(\q, \dq) + U(q) + \int_{t_{0}}^{t} \! F \cdot \frac{d\q}{d\tau} \ d\tau.
  \end{align*}
  Since the energy dynamics is captured in its entirety, this function can be used for any forced mechanical system.
\end{frame}

\begin{frame}
  \frametitle{Example: 3-Link Biped}
  \only<1>{
    \begin{columns}
      \column{1.5in}
      Dynamic Model:
      \begin{align*}
        M(q) \ddot q + H(q, \dot q) = B(q) u
      \end{align*}
      Control Law:
      \begin{align*}
        {\bar u} &= G(q) - G(\Psi(q))\\
        {\bar u}_3 &=-k_{d} (\dot \vartheta_{T}^{a})\\
        &\hspace{1.8em} -k_{p} (\vartheta_{T}^{a} - \vartheta_{T}^{d})
      \end{align*}
      \column{1.5in}
      \begin{figure}
        \centering
        \def\svgwidth{1.0\columnwidth}
        \input{figures/cg2d-3link-model.eps_latex}
        \vspace{-2em}
        \caption{3-link biped configuration.}
      \end{figure}
    \end{columns}
  }
  \only<2>{
    \begin{figure}
      \centering
      \includegraphics[width=1.0\columnwidth]{energy_conserved_cg2d_3link}
      \caption{The quantity, $E_{0}(\q, \dq) = T(\q, \dq) + V(\q) + \int_{t_{0}}^{t} F_{nc} \cdot \frac{d\q}{d\tau} d\tau$, is conserved.}
    \end{figure}
  }
  \only<3>{
    \begin{figure}
      \centering
      \includegraphics[width=1.0\columnwidth]{es_comparison_3link}
      \caption{Demonstration of energy shaping on 3-link biped.}
    \end{figure}
  }


  \only<4>{
    \begin{figure}
      \centering
      \includemedia[
        %width=1.0\columnwidth,
        %height=0.5625\columnwidth,
        width=1.0\columnwidth,
        height=0.5\columnwidth,
        addresource=3link_es.mp4,
        activate=pageopen,
        flashvars={source=3link_es.mp4&loop=true&autoPlay=true}
      ]{}{VPlayer9.swf}
      \caption{Energy shaping to stabilize to a gait from distant initial condition.}
    \end{figure}
  }

\end{frame}

\begin{frame}
  \frametitle{Multi-Domain Hybrid Systems}
  \begin{itemize}
  \item Complex hybrid systems are made by stitching together domains.
  \item Conserved energy is unique to each domain, i.e.,
    \begin{align*}
      E_0^{1} \to E_{0}^{2} \to \cdots \to E_{0}^{n-1} \to E_{0}^{n} \to E_{0}^{1} \to \cdots
    \end{align*}
  \item Each domain shapes to a specific conserved energy level.
  \end{itemize}
\end{frame}


\begin{frame}
  \frametitle{Example: 7-Link Biped with Feet}
  \only<1>{
    \blue{Appeared in {\bf Automatica}, Jun. 2014}
    \begin{columns}
      \column{1.5in}
      Dynamic Model:
      \begin{align*}
        M(q) \ddot q + H(q, \dot q) = J^{T}(q) \lambda + B(q) u
      \end{align*}
      \column{1.5in}
      \begin{figure}
        \centering
        \def\svgwidth{1.0\columnwidth}
        \input{figures/cg2d-7link-model.eps_latex}
        \vspace{-2em}
        \caption{7-link biped configuration.}
      \end{figure}
    \end{columns}
  }

  \only<2>{
    \begin{block}{Spring--Damper Controller}
      \begin{align*}
        u(\q) = -\beta e^{-\rho \, h_{\mathrm{nst}}(\q)}
      \end{align*}
      where $\beta, \rho \in \R^{+}$ and $h_{\mathrm{nst}} : \ConfigurationSpace \to \R$ is the height of the Nonstance toe.
      Used on
      \begin{itemize}
      \item Stance ankle\\
      \item Nonstance ankle\\
      \item Absolute torso angle\\
      \item Non-stance knee spring (only in double support)
      \end{itemize}
      \blue{Provides passivity-based control and toe-off.}
    \end{block}
  }

  \only<3>{
    \begin{block}{Scuffing Prevention Controller}
      \begin{align*}
        u(\q) = k_p \theta(\q) - k_d \dot \theta(\dq)
      \end{align*}
      where $k_{p}, k_{d} \in \R^{+}$. Used on
      \begin{itemize}
      \item Nonstance ankle
      \end{itemize}
      \blue{Prevents the nonstance toe from colliding with the ground prematurely.}
    \end{block}
  }

\only<4>{
  \begin{figure}
    \centering
      \includegraphics[height=.7\textheight]{domaingraph}
      \caption{The resulting gait traverses a four-domain directed graph.}
    \end{figure}
  }

  \only<5>{
    \begin{figure}
      \centering
      \includemedia[
        %width=1.0\columnwidth,
        %height=0.5625\columnwidth,
        width=1.0\columnwidth,
        height=0.5\columnwidth,
        addresource=7link_es.mp4,
        activate=pageopen,
        flashvars={source=7link_es.mp4&loop=true&autoPlay=true}
      ]{}{VPlayer9.swf}
      \caption{Energy shaping to stabilize to a gait.}
    \end{figure}
  }

  \only<6>{
    \begin{figure}
      \centering
      \includegraphics[width=1.0\columnwidth]{energy_conserved_7link}
      \caption{The quantity, $E_{0} \equiv T(q, \dot q) + V(q) + \int_{t_0}^{t} F_{nc} \cdot \frac{d\q}{d\tau} d\tau$, is conserved.}
    \end{figure}
  }

  \only<7>{
    \begin{figure}
      \centering
      \includegraphics[width=1.0\columnwidth]{energy_conserved_7link_outline}
      \caption{The quantity, $E_{0} \equiv T(q, \dot q) + V(q) + \int_{t_0}^{t} F_{nc} \cdot \frac{d\q}{d\tau} d\tau$, is conserved.}
    \end{figure}
  }
  \only<8>{
    \begin{figure}
      \centering
      \includemedia[
        %width=1.0\columnwidth,
        %height=0.5625\columnwidth,
        width=1.0\columnwidth,
        height=0.5\columnwidth,
        addresource=7link_noes.mp4,
        activate=pageopen,
        flashvars={source=7link_noes.mp4&loop=true&autoPlay=true}
      ]{}{VPlayer9.swf}
      \caption{Many steps on the limit cycle.}
    \end{figure}
  }
\end{frame}
