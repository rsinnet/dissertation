\section{Orbital Stabilization}
\showtoc

\subsection{Orbital Stabilization with Control Lyapunov Functions}
\begin{frame}
  \frametitle{Motivation}
  \begin{block}{Main Question}
    Can we use an understanding of energy exchange to improve global stability properties of periodic orbits in mechanical systems?
  \end{block}

  \begin{block}{Observations}
    \begin{itemize}
    \item Numerous control design schemes exist for stabilizing mechanical systems to periodic orbits.
    \item Some controllers produce good behavior locally but lack robustness.
    \item Periodic orbits have associated energy functions with level sets which are invariant under the orbits.
    \end{itemize}
  \end{block}
\end{frame}

\begin{frame}[t]
  \frametitle{Overview}
  \only<1>{
    \begin{block}{Setup}
      Consider the control system 
      \begin{align*}
        \dot x = f(x) + g(x) u(x).
      \end{align*}
      Assume there exists a control law ${\bar u}(x)$ which creates a limit cycle in the closed-loop dynamics,
      \begin{align*}
        {\bar f}(x) = f(x) + g(x) {\bar u}(x).
      \end{align*}
      Also assume there exists an energy function $E_{c}(x)$ which is conserved, i.e., $E(x) \equiv E_{0}$ on the limit cycle.
    \end{block}
  }

  \only<2>{
    \begin{block}{Main Idea}
      Add robustness to a periodic behavior by imposing convergence on an energy function to a level set which is known to be invariant under the system dynamics.
    \end{block}
    
    \begin{block}{Control Objective}
      Choose control input $\mu(x)$ such that $\| \mu(x) - {\bar u}(x) \|$ is minimized and $E_{c}(x(t)) \to E_0$ as $t \to \infty$.
    \end{block}

    \begin{block}{Exponential Convergence}
      To achieve exponential stabilization, $E_{c}(x(t))$ should satifisy
      \begin{align*}
        E_{c}(x(t)) \leq E_{c}(x(t_{0})) e^{\beta t} \mbox{ for } t \geq t_{0}.
      \end{align*}
    \end{block}
  }
\end{frame}

\begin{frame}
  \frametitle{Control Lyapunov Functions}
  A \blue{control Lyapunov function} $V : X \to \R$ which satisfies
  \begin{align*}
    &c_{1} \| x \|^2 \leq V(x) \leq c_{2} \|x\|^2,\\
    &\inf_{u \in U} L_f V(x) + L_g V(x) \, u + c_{3} V(x) \leq 0
  \end{align*}
  for $c_{1}, c_{2}, c_{3} > 0$ exhibits exponential convergence.
\end{frame}

\begin{frame}
  \frametitle{Energy Shaping}
  Consider a conserved energy function $E_{c}(x)$. For an exponentially stablilzing CLF, we seek $\mu(x)$ such that
  \begin{align*}
    L_f V(x) + L_g V(x) \, \mu(x) + \epsilon V(x) &\leq 0.
  \end{align*}
  Choosing $V(x) = \eta^2$ with $\eta(x) = E_{c}(x) - E_{0}$, we obtain
  \begin{align*}
    2 \eta(x) \left(L_f \eta(x) + L_g \eta(x) \, \mu(x) \right) + \epsilon \eta^2(x) \leq 0.
  \end{align*}
  We can relax this condition by augmenting the optimization space with $\delta \in \R$ and requiring
  \begin{align*}
    2 \eta(x) \left(L_f \eta(x) + L_g \eta(x) \, \mu(x) \right) + \epsilon \eta^2(x) \leq \nu \delta(x)
  \end{align*}
  where the real constant $\nu > 0$ is used to weight the relaxation.
\end{frame}

\begin{frame}[t]
  \frametitle{Quadratic Program Formulation}
  The linear form of the CLF condition suggests
  \begin{align}
    \nonumber
    (\delta^*(x), \mu(x)) = \argmin_{v = (\delta, u)}  \, & v^T \! H \, v + h^T(x) v\\
    \label{clf} \tag{clf}
    \mbox{s.t. } & \Aclf(x) \, v \leq \bclf(x)%\\
    %\label{lim} \tag{lim}
    %& \Alim v \leq \blim
  \end{align}
  where \eqref{clf} imposes the control Lyapunv function. To encode the dynamics of the system under control input ${\bar u}(x)$, select
  \begin{align*}
    H = \left(\begin{array}{c c}\nu & 0\\ 0 & I\end{array}\right), & \qquad
      h(x) = \left(\begin{array}{c} 0\\ -2 \, \bar u(x) \end{array}\right),\\
      \Aclf = \left(\begin{array}{c c}
        -1 & 2 \eta(x) \, L_{g} \eta(x)
      \end{array}\right), & \qquad
      \bclf = -\epsilon \eta^{2}(x) - 2\eta(x) \, L_{f} \eta(x).
  \end{align*}
\end{frame}


\begin{frame}
  \frametitle{Example: Compass Gait as a Shaped System}
  \only<1>{
    \begin{columns}
      \column{1.5in}
      Dynamic Model:
      \begin{align*}
        M(q) \ddot q + H(q, \dot q) = 0
      \end{align*}
      Control Law:
      \begin{align*}
        \bar u(q) &= G(q) - G(\Psi(q)).
      \end{align*}

      \column{1.5in}
      \begin{figure}
        \centering
        \def\svgwidth{1.0\columnwidth}
        \input{figures/cg2d-2link-model.eps_latex}
        \vspace{-2em}
        \caption{Compass-gait biped with Controlled Symmetries.}
      \end{figure}
    \end{columns}
  }
  \only<2>{
    \begin{figure}
      \centering
      \includegraphics[width=1.0\columnwidth]{energy_cg2d_slope_model}
      \caption{Energy of the shaped system is conserved.}
    \end{figure}    
  }
  %  \only<3>{
  %    \includemedia[
  %      width=1.0\columnwidth,
  %      height=0.5625\columnwidth,
  %      addresource=amber2d.mp4,
  %      activate=pageopen,
  %      flashvars={source=amber2d.mp4&loop=true&autoPlay=true}
  %    ]{}{}%VPlayer9.swf}
  %  }

  \only<3>{
    Energy shaping can be achieved using:
    \begin{align}
      \nonumber
      \argmin_{v = (\delta, u)}  \, & v^T \! H \, v + h^T(q, \dot q) v\\
      \label{clf} \tag{clf}
      \mbox{s.t. } & \Aclf(q, \dot q) v \leq \bclf(q, \dot q)
    \end{align}
    where
    \begin{align*}
      H = \left(\begin{array}{c c}\epsilon & 0\\ 0 & I\end{array}\right), \qquad
        h(q, \dot q) = \left(\begin{array}{c} 0\\ -2 \, \bar u(q, \dot q) \end{array}\right),
    \end{align*}
    and
    \begin{align*}
      \Aclf = \left(\begin{array}{c c}
        -1 & 2 \eta L_{g} \eta
      \end{array}\right), \qquad
      \bclf = -\epsilon \eta^{2} - 2\eta L_{f} \eta
    \end{align*}
  }
\end{frame}


\begin{frame}
  \frametitle{Nonconservative Systems}
  nconsys
\end{frame}

\begin{frame}
  \frametitle{Example: 3-Link Biped}
  \only<1>{
    \begin{columns}
      \column{1.5in}
      Dynamic Model:
      \begin{align*}
        M(q) \ddot q + H(q, \dot q) = 0
      \end{align*}
      Control Law:
      \begin{align*}
        u_1 &=-k_{d} (\dot \vartheta_{T}^{a})\\
        &\hspace{1.8em} -k_{p} (\vartheta_{T}^{a} - \vartheta_{T}^{d}),\\
        u_2 &= G(q) - G(\Psi(q)).
      \end{align*}
      \column{1.5in}
      \begin{figure}
        \centering
        \def\svgwidth{1.0\columnwidth}
        \input{figures/cg2d-3link-model.eps_latex}
        \vspace{-2em}
        \caption{3-link biped configuration.}
      \end{figure}
    \end{columns}
  }
  \only<2>{
    \begin{figure}
      \centering
      \includegraphics[width=1.0\columnwidth]{energy_conserved_cg2d_3link}
      \caption{The quantity, $E_{0} \equiv T(q, \dot q) + V(q) + \int_{0}^{t} F_{nc} \cdot dq$, is conserved.}
    \end{figure}
  }
\end{frame}
