\section{Energy Shaping}
\showtoc

\subsection{Energy Shaping with Control Lyapunov Functions}

\begin{frame}
  \frametitle{Motivation}
  \begin{block}{Main Question}
    Can we use an understanding of energy exchange to improve global stability
    properties of periodic orbits in mechanical systems?
  \end{block}

  \begin{block}{Observations}
    \begin{itemize}
    \item Numerous control design schemes exist for stabilizing mechanical
      systems to periodic orbits.
    \item Some controllers produce good behavior locally but lack robustness.
    \item Periodic orbits have associated energy functions with level sets which
      are invariant under the orbits.
    \end{itemize}
  \end{block}
\end{frame}

\begin{frame}[t]
  \frametitle{Overview}
  \only<1>{
    \begin{block}{Setup}
      Consider the control system 
      \begin{align*}
        \dx = \xf(\x) + \xg(\x) \, \uu(\x).
      \end{align*}
      Assume there exists a control law ${\bar \uu}\arx$ which creates a limit
      cycle in the closed-loop dynamics,
      \begin{align*}
        {\bar \xf}\arx = \xf\argsqdq + \xg\argsq \, {\bar \uu}\argsqdq.
      \end{align*}
      Also assume there exists an energy function $E_{c} : T\ConfigurationSpace
      \to \R$ which is conserved, i.e., $E_{c}\argsqdq \equiv E_{0}$, on the
      limit cycle.
    \end{block}
  }

  \only<2>{
    \begin{block}{Main Idea}
      Add robustness to a periodic behavior by imposing convergence on an energy
      function to a level set which is known to be invariant under the system dynamics.
    \end{block}
    
    \begin{block}{Control Objective}
      Choose control input $\mu\argsqdq$ such that $\| \mu\argsqdq - {\bar \uu}\argsqdq \|$ is minimized and $E_{c}(\q(t), \dq(t)) \to E_0$ as $t \to \infty$.
    \end{block}

    \begin{block}{Exponential Convergence}
      To achieve exponential stabilization, $E_{c}(x(t))$ should satifisy
      \begin{align*}
        E_{c}(\q(t), \dq(t)) \leq E_{c}(\q(t_{0}), \dq(t_{0})) e^{-\beta t} \mbox{ for } t \geq t_{0}, \beta > 0.
      \end{align*}
    \end{block}
  }
\end{frame}

\begin{frame}
  \frametitle{Control Lyapunov Functions}
  A \blue{control Lyapunov function} $\V : \X \to \R$ which satisfies
  \begin{align*}
    &c_{1} \nx^{2} \leq \V\arx \leq c_{2} \nx^{2},\\
    &\inf_{\uu \in \U} \Lie{\xf}\V\arx + \Lie{\xg}\V\arx \, \uu + c_{3} \V\arx \leq 0
  \end{align*}
  for $c_{1}, c_{2}, c_{3} > 0$ exhibits exponential convergence. If the above
  are satisfied, then it is also true that
  \begin{align*}
    \left\| \pd{\V\arx}{\x} \right\| \leq c_{4} \nx.
  \end{align*}
\end{frame}


\begin{frame}
  \frametitle{Rapidly Exponentially Stabilizing Control Lyapunov Functions}
  A \blue{rapidly exponentially stability control Lyapunov function (RES--CLF)}
  $\Ve : \X \to \Rnn$ satisfies
  \begin{align*}
    &c_{1} \nx^{2} \leq \Ve\arx \leq \frac{c_{2}}{\resclfparam^{2}} \nx^{2},\\
    &\inf_{\uu \in \U} \Lie{\xf}\Ve\arx + \Lie{\xg}\Ve\arx \, \uu +
    \frac{c_{3}}{\resclfparam} \Ve\arx \leq 0
  \end{align*}
  for $c_{1}, c_{2}, c_{3} > 0$ exhibits exponential convergence. If the above
  are satisfied, then it is also true that
  \begin{align*}
    \left\| \pd{\Ve\arx}{\x} \right\| \leq c_{4} \nx.
  \end{align*}
\end{frame}

\begin{frame}
  \frametitle{Energy Shaping}
  Consider a conserved energy function $E_{c}\arx$ and define a candidate
  control Lyapunov function:
  \begin{align*}
    V\arx = \frac{1}{2} \left(\Ec\arx - \Eref\right).
  \end{align*}
  For an exponentially stabilizing CLF, we seek $\mu\arx$ such that
  \begin{align*}
    \Lie{\xf} V\arx + \Lie{\xg} \V\arx \, \mu\arx + \epsilon \V\arx &\leq 0.
  \end{align*}
  % We can relax this condition by augmenting the optimization space with $\delta \in \R$ and requiring
  %\begin{align*}
  %  2 \eta\arx \left(\Lie{\xf} \eta\arx + \Lie{\xg} \eta\arx \, \mu\arx \right) + \epsilon \eta^{2}\arx \leq \delta\arx.
  %\end{align*}
\end{frame}

\begin{frame}[t]
  \frametitle{Quadratic Program Formulation}
  The linear form of the CLF condition suggests
  \begin{align}
    \nonumber
    \mueps\arx = \argmin_{\uu \in \R^{n}}  \, & \uu^T \uu\\
    \label{eq:clf} \tag{clf}
    \mbox{s.t. } & \Aclf\arx \, \uu \leq \bclf\arx%\\
    %\label{lim} \tag{lim}
    %& \Alim v \leq \blim
  \end{align}
  where \eqref{eq:clf} imposes the control Lyapunov function. To encode the dynamics of the system, select
  \begin{align*}
    \Aclf = \Lie{\xg}\Ve\arx
    \quad
    \bclf = -\Lie{\xf}\Ve\arx - \frac{c_{3}}{\resclfparam} \Ve\arx.
  \end{align*}
\end{frame}

\begin{frame}
  Applying the energy shaping controller gives the closed-loop hybrid system
  \begin{align*}
    \HS = \left\{
      \begin{array}{l l}
        \dx = \xf\arx + \xg\arx \, \mueps, & \x \in \D \setminus \S,\\
        \xp = \Delta(\xm), & \x \in \S.
      \end{array}\right.
  \end{align*}
  
\end{frame}

\subsection{Energy Shaping Theorem}

\begin{frame}
  \frametitle{Main Theorem}

  \begin{theorem}
    Given an exponentially-stable cycle in a hybrid system, application
    of the energy shaping controller results in the closed-loop hybrid system
    \begin{align*}
      \HS_{\resclfparam} = \left\{
      \begin{array}{l l}
        \dx = \xf\arx + \xg\arx \, \mueps\arx, & \x \in \D \setminus \S,\\
        \xp = \Delta(\xm), & \x \in \S,
      \end{array}\right.
  \end{align*}
  which is exponentially stable for small enough $\resclfparam$.
  
  \end{theorem}
\end{frame}

\begin{frame}
  \frametitle{Zero Dynamics Formulation}
  Construct a change of coordinates, splitting up the system into two sets of coordinates:
  \begin{align*}
    \dot \zdx &= f\argsxz + g\argsxz \, \uu,\\
    \dot \zdz &= q\argsxz + w\argsxz \, \uu.
  \end{align*}
  The vector fields $f$, $g$, $q$, and $w$ are assumed to be locally Lipschitz continuous.
\end{frame}

\begin{frame}
  \frametitle{Energy-Based Coordinate Change}
  For mechanical systems with coordinates $\x = \argsqdq \in T\Q$, construct the transformation $\Phi\argsqdq := \argsxz$ where
  \begin{align*}
    \zdx &= \Ec\argsqdq - \Eref,\\
    \zdz &= \xrem,
  \end{align*}
  where $n$ is the size of the configuration space, $\Q$. The fixed point of the
  hybrid system is chosen to occur at $\argsxz = \xzst$ such that $\Delta\xzst =
  \argszero$.
\end{frame}

\begin{frame}
  \frametitle{Validity of Transformation}
  The coordinate change $\Phi\argsqdq$ is valid if it is locally diffeomorphic around the orbit
  $\orbit$, i.e., if
  \begin{align}
    \pd{\Psi(\q, \dq)}{(\q, \dq)} =
    \left(\begin{array}{c c}
        I & \boldzero_{2n-1 \times 1}\\
        \pd{E(\q, \dq)}{\xrem} & \pd{E(\q, \dq)}{\dq_{n}}
      \end{array}\right)
  \end{align}
  % 
  has full rank, which occurs when
  \begin{align*}
    \det \Psi\argsqdq = \pd{E(\q, \dq)}{\dq_{n}} \ne 0.
  \end{align*}
\end{frame}
  
\begin{frame}
  \frametitle{Sketch of Proof}
  Define a composite Lyapunov function,
  \begin{align*}
    \VP\argsxz = \Vn\argsxz + \sigma \Vex(\zdx),
  \end{align*}
  where
  \begin{itemize}
  \item $\Vn$ : Lyapunov function guaranteed by stability of nominal system
    (converse Lyapunov theorem)
  \item $\Vex$ : Lyapunov function for energy shaping control law
  \end{itemize}
  with scaling parameter $\sigma$. This is a discrete Lyapunov function that is
  valid on the guard.

\end{frame}

\begin{frame}
  \frametitle{Sketch of Proof}  
  and show that
  \begin{align*}
    \lefteqn{\VP(\Pe\argsxz) - \VP\argsxz =}\\
    &&\Vn(\Pe\argsxz) - \Vn\argsxz + \sigma(\Vex(\Pe\argsxz) - \Vex(\zdx)).
  \end{align*}
\end{frame}
