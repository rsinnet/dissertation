\section{Energy Shaping}
\showtoc

\subsection{Energy Shaping with Control Lyapunov Functions}

\begin{frame}[t]
  \frametitle{Motivation}
  \begin{block}{Main Question}
    Can we use an understanding of energy exchange to improve robustness  of
    periodic orbits in hybrid mechanical systems?
  \end{block}

  \begin{block}{Observations}
    \begin{itemize}
    \item Numerous control design schemes exist for stabilizing hybrid mechanical
      systems to periodic orbits.
    \item Some controllers produce good behavior locally but lack robustness.
    \item Periodic orbits have associated energy functions with level sets which
      are invariant under the orbits.
    \end{itemize}
  \end{block}
\end{frame}

\begin{frame}[t]
  \frametitle{Overview}
  \begin{block}{Main Idea}
    Add robustness to a periodic behavior by imposing convergence on a conserved
    energy function, $\Ec : \x \to \R$, to a level set which is known to be
    invariant under the system dynamics.
  \end{block}
  
  \begin{block}{Control Objective}
    Choose control input $\mu\arx$ such that $\| \mu\arx \|$ is minimized and $\Ec\arxt \to \Eref$ as $t \to \infty$.
  \end{block}

  \begin{block}{Exponential Convergence}
    To achieve exponential stabilization, $\Ec\arxt$ should satisfy\vspace{-.4em}
    \begin{align*}
      \Ec\arxt \leq \Ec\arxzero e^{-\beta t} \mbox{ for } t \geq 0, \beta > 0.
    \end{align*}
  \end{block}
\end{frame}

\begin{frame}[t]
  \frametitle{Conserved Energy Functions}
  \only<1> {
    \begin{block}{Conservative Systems}
      A conservative system with state space $\x = \argsqdq$ is modeled as
      \begin{align*}
        \HCSbar = \left\{
          \begin{array}{l l}
            \left.\begin{array}{r c l}
                \hspace{.58em}\dx &=& \xfbar\argsqdq + \xgbar\argsq \, \uu
              \end{array}\right\}  & \mbox{if } \argsqdq \in \D \setminus \S,\\
            \left. \begin{array}{r c l}
                \qp &=& \Deltaq\argsqm\\
                \dqp &=& \Deltadq\argsqdqm
              \end{array} \right\} & \mbox{if } \argsqdq \in \S,
          \end{array}\right.
      \end{align*}
      where $\xfbar\argsqdq := \xf\argsqdq$ and $\xgbar\argsq := \xg\argsq$ for
      notational clarity. Total energy is conserved through the continuous
      dynamics, i.e.,
      \begin{align*}
        \Ec\argsqdq := T\argsqdq + U\argsq.
      \end{align*}
    \end{block}
  }

  \only<2> {
    \begin{block}{Nonconservative Systems}
      To properly handle the flow of energy due to $\vv\argsqdq$, define a storage
      function, $\W$, which obeys the differential equation
      \begin{align*}
        d\W = \vfR\argsqdq \, dt = \left( \B\argsq \, \vv\argsqdq \right)^{T}
        \frac{d\q}{dt}\art \, dt.
      \end{align*}
      Using $\W$, augment the state space, i.e., $\x := (\q, \dq, \W)$, and the vector
      fields (subsuming $\vv\argsqdq$ under $\xfbar\argsqdq$), i.e, 
      \begin{align*}
        \xfbar\argsqdq := \left(\begin{array}{c}
            \xf\argsqdq + \xg\argsq \, \vv\argsqdq\\
            \vfR\argsqdq
          \end{array}\right), &&
        \xgbar\argsq := \left(\begin{array}{c}
            \xg\argsq\\
            \boldzero
          \end{array}\right).
      \end{align*}
    \end{block}
  }
  \only<3> {
    \begin{block}{Nonconservative Systems}
      Use the augmented state to define the hybrid control system
      \begin{align*}
        \HCSbar = \left\{
          \begin{array}{l l}
            \left.\begin{array}{r c l}
                \hspace{1.15em}\dx &=& \xfbar\argsqdq + \xgbar\argsq \, \uu
              \end{array}\right\}  & \mbox{if } \argsqdq \in \D \setminus \S,\\
            \left. \begin{array}{r c l}
                \qp &=& \Deltaq\argsqm\\
                \dqp &=& \Deltadq\argsqdqm\\
                \Wp &=& \DeltaW = 0
              \end{array} \right\} & \mbox{if } \argsqdq \in \S.
          \end{array}\right.
      \end{align*}
      For such a system, the following quantity is conserved:
      \begin{align*}
        \Ec\argsqdqW := T\argsqdq + U\argsq - W.
      \end{align*}
    \end{block}
  }
\end{frame}

\begin{frame}[t]
  \frametitle{Rapidly Exponentially Stabilizing CLFs}
  A \blue{rapidly exponentially stability control Lyapunov function (RES--CLF)}
  $\Ve : \X \to \Rnn$ satisfies
  \begin{align*}
    &c_{1} \nx^{2} \leq \Ve\arx \leq \frac{c_{2}}{\resclfparam^{2}} \nx^{2},\\
    &\inf_{\uu \in \U} \Lie{\xfbar}\Ve\arx + \Lie{\xgbar}\Ve\arx \, \uu +
    \frac{c_{3}}{\resclfparam} \Ve\arx \leq 0
  \end{align*}
  for $c_{1}, c_{2}, c_{3} > 0$ exhibits exponential convergence. If the above
  are satisfied, then it is also true that
  \begin{align*}
    \left\| \pd{\Ve\arx}{\x} \right\| \leq c_{4} \nx.
  \end{align*}
\end{frame}

\begin{frame}[t]
  \frametitle{Energy Shaping}
  Consider a conserved energy function $\Ec\arx$ on a hybrid control system
  $\HCSbar$ which has a periodic orbit $\orbit$ and define a Lyapunov candidate:
  \begin{align*}
    V\arx = \frac{1}{2} \left(\Ec\arx - \Eref\right)^{2},
  \end{align*}
  with $\Eref$ the constant energy level of the system on the orbit
  $\orbit$. For a RES--CLF, we seek a feedback control law, $\mu\arx$, such that
  \begin{align*}
    \Lie{\xfbar} V\arx + \Lie{\xgbar} \V\arx \, \mu\arx + \epsilon \V\arx &\leq 0.
  \end{align*}
\end{frame}

\begin{frame}[t]
  \frametitle{Quadratic Program Formulation}
  The linear form of the RES--CLF condition suggests
  \begin{align}
    \nonumber
    \mueps\arx = \argmin_{\uu \in \R^{n}}  \, & \uu^T \uu,\\
    \mbox{s.t. } & \Aclf\arx \, \uu \leq \bclf\arx
  \end{align}
  which encodes the dynamics of the system. This controller imposes exponential
  stabilization of the energy as defined by the RES--CLF.
\end{frame}

\begin{frame}[t]
  \frametitle{Main Theorem}
  \begin{block}{Theorem [Energy Shaping]}
    Given an exponentially-stable cycle in a hybrid system, application
    of the energy shaping controller results in the closed-loop hybrid system
    \begin{align*}
      \HS_{\resclfparam} = \left\{
        \begin{array}{r c l l}
          \dx &=& \xfbar\arx + \xgbar\arx \, \mueps\arx, & \x \in \D \setminus \S,\\
          \xp &=& \Delta(\xm), & \x \in \S,
        \end{array}\right.
    \end{align*}
    which is exponentially stable about the hybrid periodic orbit $\orbit$ for
    large enough $\resclfparam$.
  \end{block}
\end{frame}

\subsection{Sketch of Proof}

\begin{frame}[t]
  \frametitle{Overview of Proof}
  \begin{block}{Sketch of Proof [Energy Shaping]}
    \begin{enumerate}
    \item Transform the coordinates into a more intuitive form.
    \item Define a discrete Lyapunov candidate function valid on the \Poincare{} map.
    \item Show the conditions for stability through bounding arguments.
    \end{enumerate}
  \end{block}
\end{frame}

\begin{frame}[t]
  \frametitle{Zero Dynamics Formulation}
  Construct a change of coordinates, splitting up the system into two sets of coordinates:
  \begin{align*}
    \dot \zdx &= f\argsxz + g\argsxz \, \uu,\\
    \dot \zdz &= q\argsxz + w\argsxz \, \uu.
  \end{align*}
  The vector fields $f$, $g$, $q$, and $w$ are assumed to be locally Lipschitz
  continuous. For simplicity, define
  \begin{align*}
    \bigF\argsxz = \left(\begin{array}{c}
        f\argsxz\\
        q\argsxz
      \end{array}\right),&&
    \bigG \argsxz = \left(\begin{array}{c}
        g\argsxz\\
        w\argsxz
      \end{array}\right).
  \end{align*}
\end{frame}

\begin{frame}[t]
  \frametitle{Energy-Based Coordinate Change}
  \only<1> {
    \begin{block}{Conservative Systems}
      Using the state $\x = \argsqdq$, construct the transformation $\xform\argsqdqW := \argsxz$ where
      \begin{align*}
        \zdx &= \Ec\argsqdq - \Eref,\\
        \zdz &= \xrem,
      \end{align*}
    \end{block}
  }

  \only<2> {
    \begin{block}{Nonconservative Systems}
      Using the state $\x = \argsqdqW$, construct the transformation $\xform\argsqdqW := \argsxz$ where
      \begin{align*}
        \zdx &= \Ec\argsqdqW - \Eref,\\
        \zdz &= \xremW,
      \end{align*}
    \end{block}
  }
  where $n$ is the size of the configuration space, $\Q$. The fixed point of the
  hybrid system is chosen to occur at $\argsxz = \xzst$ such that $\Delta\xzst =
  \argszero$.
\end{frame}

\begin{frame}[t]
  \frametitle{Validity of Transformation}
  \only<1> {
    \begin{block}{Conservative Systems}
      The coordinate change $\xform\arx = \xform\argsqdq$ is valid if it is locally
      diffeomorphic around the orbit $\orbit$, i.e., if
      \begin{align*}
        \pd{\xform\argsqdq}{\argsqdq} =
        \left(\begin{array}{c c}
            I_{2n-1} & \boldzero_{2n-1 \times 1}\\
            \pd{\Ec(\q, \dq)}{\xrem} & \pd{\Ec(\q, \dq)}{\dq_{n}}
          \end{array}\right)
      \end{align*}
      % 
      has full rank, which occurs when
      \begin{align*}
        \det\left(\pd{\xform\argsqdq}{\argsqdq}\right) = \pd{\Ec\argsqdq}{\dq_{n}} \ne 0.
      \end{align*}
    \end{block}
  }
  
  \only<2> {
    \begin{block}{Nonconservative Systems}
      The coordinate change $\xform\arx = \xform\argsqdqW$ is valid if it is locally
      diffeomorphic around the orbit $\orbit$, i.e., if
      \begin{align*}
        \pd{\xform\argsqdqW}{\argsqdqW} =
        \left(\begin{array}{c c c}
            I_{2n-1} & \boldzero_{2n-1 \times 1} & \boldzero_{2n-1 \times 1}\\
            \pd{\E\argsqdqW}{\xrem} & \pd{\Ec\argsqdqW}{\dq_{n}} & 1\\
            \boldzero_{1 \times 2n-1} & 0 & 1
          \end{array}\right)
      \end{align*}
      % 
      has full rank, which occurs when
      \begin{align*}
        \det \pd{\xform\argsqdqW}{\argsqdqW} = \pd{\Ec\argsqdqW}{\dq_{n}} \ne 0.
      \end{align*}
    \end{block}
  }
\end{frame}

\begin{frame}[t]
  \frametitle{Flows of Continuous Dynamical Systems}
  The flow of an ODE expressed in the variables $\argsxz$ is given
  by
  \begin{align*}
    \flowt\argsxz = \int_{0}^{t} \left[ \bigF\argsxztau + \bigG\argsxztau \,
      \uu\argsxztau \right] \, d\tau
  \end{align*}
  for some feedback control law $\uu : \zdX \times \zdZ \to \U$. For some stable
  tube $\stabletube$ around the orbit $\orbit$, it holds that the vector fields are bounded:
  \begin{align*}
    \supF & := \sup \left\{ \left\| \bigF\argsxz \right\| : \argsxz \in
      \stabletube \right\},\\
    \lambdamaxG &:= \sup \left\{\lambdamax\bigG\argsxz : \argsxz \in \stabletube
    \right\},\\
    \lambdaminG &:= \inf \left\{\lambdamin\bigG\argsxz : \argsxz \in \stabletube
    \right\}.
  \end{align*}
\end{frame}


\begin{frame}[t]
  \frametitle{Definition of the \Poincare{} Map}
  \only<1> {
    The \blue{\Poincare{} first return map} takes a point on the guard, applies
    the reset map and then integrates forward until the guard is reached
    again. For the shaped system, $\Pe : \S \to \S$, is defined by
    \begin{align*}
      \Pe\argsxz = \floweps_{\TIeDelta}\argsDeltaxz,
    \end{align*}
    where $\TIe : \zdX \times \zdZ \to \Rnn$ is the \tti{} function which is
    defined by
    \begin{align*}
      \TIe\argsxz = \inf \{ t \geq 0 : h(\flowepst\argsxz) \}
    \end{align*}
    and is locally Lipschitz continuous by the implicit function theorem.
  }

  \only<2> {
    \begin{figure}    
      \centering
      \def\svgwidth{.78\columnwidth}
      \input{../figs/poincare_return_map.eps_latex}
      \caption{The \Poincare{} map is defined by the \tti{} function and maps
        from states on the \Poincare{} section $\S$ back to the section.}
    \end{figure}
  }
\end{frame}

\begin{frame}[t]
  \frametitle{Equivalence of Hybrid Periodic Orbits}
  \begin{lemma}[Equivalence of Orbits]
    Applying the energy shaping controller to the hybrid control system $\HCSbar$
    results in the closed-loop hybrid system $\HSepsbar$ that demonstrates a
    periodic orbit which is identical to the unshaped system $\HSbar$.
  \end{lemma}
  \begin{block}{Proof}
    The energy of states $\xo \in \orbit$ is a constant,
    $\E(\xo) \equiv \Eref$. By the choice of Lyapunov function,
    $\V(\xo) \equiv 0$. In addition, $\dV(\xo) \equiv 0$ since
    the system is conservative. And since the solution to the optimization
    problem has cost $\uu^{T}(\xo) \uu(\xo) \equiv 0$, the
    control effort is also zero. Hence the orbits are equivalent.
  \end{block}
\end{frame}

\begin{frame}[t]
  \frametitle{Boundedness of \TtI{} Function}
  \only<1> {
    The difference in solutions between the shaped and unshaped systems can be
    bounded by examining the difference in solutions:
    \begin{lemma}[Boundedness of \TtI{}]
      For the hybrid systems $\HSbar$ and $\HSepsbar$,
      \begin{eqnarray}
        \nonumber
        &| \TIe(\Deltaxz) - \TI(\Deltaxz) | \leq \ATIe \nzdxzst,
      \end{eqnarray}
      where $\limeps \ATIe = 0$.
    \end{lemma}
  }
  \only<2> {
    \begin{block}{Proof (Boundedness of \TtI{})}
      Construct an auxiliary \tti{} function $\TB$ which is locally Lipschitz
      continuous and independent of $\resclfparam$:
      \begin{align*}
        \TB(\eta, \zdx, \zdz) = \inf\{t \geq 0 : h(\eta + \flowt(\Delta(0, \zdzst))) = 0\}.
      \end{align*}
      By construction,
      \begin{align*}
        \TB(0, \zdx, \zdz) = \TI\argsxz.
      \end{align*}
    \end{block}
  }

  \only<3> {
    \begin{block}{Proof (Boundedness of \TtI{})}
      Let $\argsxzti{1}$ and $\argsxzti{2}$ satisfy\vspace{-.4em}
      \begin{align*}
        \argsDxzti{1} = \floweps\argsxzti{1}, && \argsDxzti{2} =
        \flow\argsxzti{2},
      \end{align*}
      with initial conditions $\argsxzizero{1} = \argsxzizero{2} = \Delta(0,
      \zdzst)$. Define
      \begin{align*}
        \etaeps = \left.\argsxzti{1}\right|_{t = \TIe\argsxz} -
        \left.\argsxzti{2}\right|_{t = \TIe\argsxz}
      \end{align*}
      and as a result,\vspace{-.4em}
      \begin{align*}
        \TIe\argsxz = \TB(\etaeps, \zdx, \zdz).
      \end{align*}

    \end{block}
  }

  \only<4> {
    \begin{block}{Proof (Boundedness of \TtI{})}
      Examining the explicit solution given by the min-norm control law,
      \begin{align*}
        \mueps\argsxz = - \frac{\frac{c_{3}}{\resclfparam} \Ve\argsxz \bigG^{T}\argsxz \left( \pd{V_{e}\argsxz}{\argsxz} \right)^{T}}{\pd{V_{e}\argsxz}{\argsxz} \bigG\argsxz \bigG^{T}\argsxz \left( \pd{V_{e}\argsxz}{\argsxz} \right)^{T}},
      \end{align*}
      a bound can be established:
      \begin{align*}
        \| \mueps\argsxz \| \leq \frac{c_{2}c_{3}}{\resclfparam^{3}} \frac{\lambdamaxG}{\lambdaminG} \nzdx.
      \end{align*}
    \end{block}
  }

  \only<5> {
    \begin{block}{Proof (Boundedness of \TtI{})}
      The difference between flows can then be bounded:
      \begin{align*}
        \|\etaeps\| &\leq \int_{0}^{\TIe\argsxz} \! \left\| \bigG\argsxztau \,
          \mueps\argsxztau \, \right\| d\tau\\
        &\leq \LDelta \frac{c_{2}^{\frac{3}{2}}
          c_{3}^{2}}{2c_{1}^{\frac{1}{2}}\resclfparam^{5}}
        \frac{\lambdamaxG^{2}}{\lambdaminG^{2}}  \, \left(1 -
          e^{-\frac{c_{3}}{2\resclfparam} 1.1 \Tst}\right) \nzdxzst
      \end{align*}
    \end{block}
  }
  
  \only<6> {
    \begin{block}{Proof (Boundedness of \TtI{})}
      Because $\TB$ is Lipschitz continuous, it follows that
      \begin{align*}
        | \TIe\argsxz - \TI\argsxz | &= | \TB(\etaeps, \zdx, \zdz) - \TB(0, \zdx, \zdz)|\\
        &\leq \LB \| \etaeps \|.
      \end{align*}
      Using the bound for $\etaeps$ leads to the conclusion that
      \begin{align*}
        | \TIe(\Deltaxz) - \TI(\Deltaxz) | \leq \ATIe \nzdxzst.
      \end{align*}
      with $\limeps \ATIe = 0$. \hfill \qedsymbol
    \end{block}
  }
\end{frame}

\begin{frame}[t]
  \frametitle{Boundedness of \Poincare{} Maps}
  \only<1> {
    The difference in \Poincare{} maps of the unshaped and shaped systems can be
    bounded by examining the difference in solutions:
    \begin{lemma}[Boundedness of \Poincare{} Maps]
      For the hybrid systems $\HSbar$ and $\HSepsbar$,
      \begin{eqnarray}
        \nonumber
        &\| \Pe\argsxz - \P\argsxz \| \leq \Ae \nzdxzst,
      \end{eqnarray}
      where  $\limeps\Ae = 0$.
    \end{lemma}
  }

  \only<2> {
    \begin{block}{Proof (Boundedness of \Poincare{} Maps)}
      \begin{align*}
        \lefteqn{\left\| \Pe\argsxz - \P\argsxz \right\| \leq}\\
        & \hspace{2em} \int_{\TIDelta}^{\TIeDelta} \hspace{-2.5em} \left\| \bigF\argsxztau \right\| \, d\tau\\
        & \hspace{4em} \mbox{} + \int_{0}^{\TIeDelta} \hspace{-2.5em} \left\| \bigG\argsxztau \, \mueps\argsxztau \, \right\| d\tau.
      \end{align*}
    \end{block}
  }

  \only<3> {
    \begin{block}{Proof (Boundedness of \Poincare{} Maps)}
      Using the explicit solution given by the min-norm controller and the bounds
      established on the difference in \tti{} functions, the bounds can be
      evaluated and the result is a bound of the form
      \begin{align*}
        \left\| \Pe\argsxz - \P\argsxz \right\| \leq \Ae \! \nzdxzst,
      \end{align*}
      where  $\limeps\Ae = 0$. \hfill \qedsymbol
    \end{block}
  }
\end{frame}

\begin{frame}[t]
  \frametitle{Proof of Energy Shaping Theorem}
  \only<1> {
    Define a composite Lyapunov function,
    \begin{align*}
      \VP\argsxz = \Vn\argsxz + \sigma \Vex(\zdx),
    \end{align*}
    where
    \begin{itemize}
    \item $\Vn$ : Lyapunov function guaranteed by stability of the nominal system
      (converse Lyapunov theorem)
    \item $\Vex$ : Lyapunov function for energy shaping control law
    \end{itemize}
    with scaling parameter $\sigma$. This is a discrete Lyapunov function that
    is valid on the guard.
  }
  
  \only<2> {
    Exponential stability of the hybrid system $\HS$ is guaranteed by the
    discrete Lyapunov theorem if
    \begin{eqnarray*}
      &r_{1} \nzdxzst^{2} \leq \VP\argsxz \leq r_{2} \nzdxzst^{2}\\
      &\VP(\Pe\argsxz) - \VP\argsxz \leq r_{3} \nzdxzst^{2}
    \end{eqnarray*}
    for some $r_{1}, r_{2}, r_{3} \in \Rnn$.
  }
\end{frame}
