\section{Energy Shaping}
\showtoc

\subsection{Energy Shaping with Control Lyapunov Functions}

\begin{frame}
  \frametitle{Motivation}
  \begin{block}{Main Question}
    Can we use an understanding of energy exchange to improve global stability properties of periodic orbits in mechanical systems?
  \end{block}

  \begin{block}{Observations}
    \begin{itemize}
    \item Numerous control design schemes exist for stabilizing mechanical systems to periodic orbits.
    \item Some controllers produce good behavior locally but lack robustness.
    \item Periodic orbits have associated energy functions with level sets which are invariant under the orbits.
    \end{itemize}
  \end{block}
\end{frame}

\begin{frame}[t]
  \frametitle{Overview}
  \only<1>{
    \begin{block}{Setup}
      Consider the control system 
      \begin{align*}
        \dx = \xf(\x) + \xg(\x) \, \uu(\x).
      \end{align*}
      Assume there exists a control law ${\bar \uu}\arx$ which creates a limit cycle in the closed-loop dynamics,
      \begin{align*}
        {\bar \xf}\arx = \xf\argsqdq + \xg\argsq \, {\bar \uu}\argsqdq.
      \end{align*}
      Also assume there exists an energy function $E_{c} : T\ConfigurationSpace \to \R$ which is conserved, i.e., $E_{c}\argsqdq \equiv E_{0}$, on the limit cycle.
    \end{block}
  }

  \only<2>{
    \begin{block}{Main Idea}
      Add robustness to a periodic behavior by imposing convergence on an energy function to a level set which is known to be invariant under the system dynamics.
    \end{block}
    
    \begin{block}{Control Objective}
      Choose control input $\mu\argsqdq$ such that $\| \mu\argsqdq - {\bar \uu}\argsqdq \|$ is minimized and $E_{c}(\q(t), \dq(t)) \to E_0$ as $t \to \infty$.
    \end{block}

    \begin{block}{Exponential Convergence}
      To achieve exponential stabilization, $E_{c}(x(t))$ should satifisy
      \begin{align*}
        E_{c}(\q(t), \dq(t)) \leq E_{c}(\q(t_{0}), \dq(t_{0})) e^{-\beta t} \mbox{ for } t \geq t_{0}, \beta > 0.
      \end{align*}
    \end{block}
  }
\end{frame}

\begin{frame}
  \frametitle{Control Lyapunov Functions}
  A \blue{control Lyapunov function} $\V : \X \to \R$ which satisfies
  \begin{align*}
    &c_{1} \nx^{2} \leq \V(\x) \leq c_{2} \nx^{2},\\
    &\inf_{\uu \in \U} \Lie{\xf} \V(\x) + \Lie{\xg} V\arx \, \uu + c_{3} \V\arx \leq 0
  \end{align*}
  for $c_{1}, c_{2}, c_{3} > 0$ exhibits exponential convergence.
\end{frame}

\begin{frame}
  \frametitle{Energy Shaping}
  Consider a conserved energy function $E_{c}\arx$. For an exponentially stabilizing CLF, we seek $\mu\arx$ such that
  \begin{align*}
    \Lie{\xf} V\arx + \Lie{\xg} \V\arx \, \mu\arx + \epsilon \V\arx &\leq 0.
  \end{align*}
  Choosing $V\arx = \eta^{2}$ with $\eta\arx = E_{c}\arx - E_{0}$, we obtain
  \begin{align*}
    2 \eta\arx \left(\Lie{\xf} \eta\arx + \Lie{\xg} \eta\arx \, \mu\arx \right) + \epsilon \eta^{2}\arx \leq 0.
  \end{align*}
  %We can relax this condition by augmenting the optimization space with $\delta \in \R$ and requiring
  %\begin{align*}
  %  2 \eta\arx \left(\Lie{\xf} \eta\arx + \Lie{\xg} \eta\arx \, \mu\arx \right) + \epsilon \eta^{2}\arx \leq \delta\arx.
  %\end{align*}
\end{frame}

\begin{frame}[t]
  \frametitle{Quadratic Program Formulation}
  The linear form of the CLF condition suggests
  \begin{align}
    \nonumber
    \mueps\arx = \argmin_{\uu \in \R^{n}}  \, & \uu^T \uu\\
    \label{eq:clf} \tag{clf}
    \mbox{s.t. } & \Aclf(x) \, \uu \leq \bclf(x)%\\
    %\label{lim} \tag{lim}
    %& \Alim v \leq \blim
  \end{align}
  where \eqref{eq:clf} imposes the control Lyapunv function. To encode the dynamics of the system under control input ${\bar \uu}\arx$, select
  \begin{align*}
    \Aclf = 2 \eta\arx \, \Lie{\xg} \eta\arx
    \quad
    \bclf = -\epsilon \eta^{2}\arx - 2\eta\arx \, \Lie{\xf} \eta\arx.
  \end{align*}
\end{frame}

\subsection{HSYS}

\begin{frame}
  Energy shaping operates on hybrid control systems of the form
  \begin{align*}
    \HCS = \left\{
      \begin{array}{l l}
        \dx = \xf\arx + \xg\arx \, \uu, & \x \in \D \setminus \S,\\
        \xp = \Delta(\xm), & \x \in \S,
      \end{array}\right.
  \end{align*}
\end{frame}

\begin{frame}
  \frametitle{Main Theorem}

  \begin{theorem}
    Given an exponentially-stable cycle in a hybrid system, application
    of the energy shaping controller results in the closed-loop hybrid system
    \begin{align*}
      \HS_{\resclfparam} = \left\{
      \begin{array}{l l}
        \dx = \xf\arx + \xg\arx \, \mueps\arx, & \x \in \D \setminus \S,\\
        \xp = \Delta(\xm), & \x \in \S,
      \end{array}\right.
  \end{align*}
  which is exponentially stable.
  \end{theorem}
\end{frame}

\begin{frame}
  \frametitle{Sketch of Proof}
  Define a composite Lyapunov function,
  \begin{align*}
    \VP\argsxz = \Vn\argsxz + \sigma \Vex(\zdx),
  \end{align*}
  where
  \begin{itemize}
  \item $\Vn$ : Lyapunov function guaranteed by stability of nominal system
    (converse Lyapunov theorem)
  \item $\Vex$ : Lyapunov function for energy shaping control law
  \end{itemize}
  with scaling parameter $\sigma$ and show that
  \begin{align*}
    \lefteqn{\VP(\Pe\argsxz) - \VP\argsxz =}\\
    &&\Vn(\Pe\argsxz) - \Vn\argsxz + \sigma(\Vex(\Pe\argsxz) - \Vex(\zdx)).
  \end{align*}
\end{frame}
